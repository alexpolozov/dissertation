\section{Feedback-Based Synthesis}
\label{sec:interactive:feedback}

As we discussed above, an end user often does not have a clear understanding of the full spec, and may suffer
significant cognitive load with this task.
At every iteration, the current candidate program set $\vsa_i$ contains thousands of ambiguous programs, and humans
struggle with reasoning about possible ambiguities in intent specification.
In contrast, the synthesis system can analyze ambiguities in $\vsa_i$ and derive the most efficient way to resolve them
by proactively soliciting concrete knowledge from the user.
This observation introduces a new actor in the traditional learner-user interaction process, which we call \emph{the
hypothesizer}.

\Cref{fig:feedback:hypothesizer} shows our envisioned workflow.
Formally, any \emph{constraint type} used in PBE (e.g. example, prefix, output type) states a \emph{property} on a
subset of the DSL.
Given a program set $\vsa_i$, the hypothesizer deduces possible properties that best disambiguate among programs in
$\vsa_i$.
Any such property is convertible to a Boolean or multiple-choice question $q$, which the hypothesizer asks the user.
Any response $r$ for $q$ is convertible to a concrete constraint $\constraint$, which begins a new iteration of
synthesis.

Such \emph{feedback-based} interaction has several major benefits.
First, it reduces the cognitive load on the user: instead of analyzing the program's behavior, she only answers concrete
questions.
Second, it significantly reduces the number of synthesis iterations thanks to the hypothesizer's insight into the
program set $\vsa$ and its choice of disambiguating questions.
Finally, feedback and proactiveness increases the user's confidence and trust in the system.

\tikzstyle{flowchart} = [semithick, align=center, >=stealth', every path/.style={->, font=\small}]
\tikzstyle{actor} = [draw, rounded corners, inner sep=6pt, node distance=2cm, font=\sffamily, outer sep=3pt,
                     fill=blue!20]
\tikzstyle{data} = [inner sep=5pt]
\tikzstyle{empty} = [inner sep=0pt, outer sep=0pt]
\tikzstyle{pass} = [densely dashed, thin, >=spaced open triangle 45, color=gray]
\tikzstyle{edgelabel} = [sloped, anchor=center, above]

\begin{figure}[t]
    \centering
    \begin{tikzpicture}[flowchart]
        \node [actor] (learner) {Learner};
        \node [actor, right=3cm of learner] (user) {User};
        \path (learner) to coordinate[midway] (mid) (user);
        \node [actor, above=3.0cm of mid] (analyzer) {Hypothesizer};
        \node [data, left=of learner] (ranking) {Ranking function $h$};
        \draw (ranking) to (learner);
        \draw (learner) to ["VSA $\vsa$" {edgelabel, pos=0.33}, bend left=20]
            node[empty, pos=0.85] (lstart) {}
            (analyzer);
        \draw (user) to["spec $\spec$"]
            node[empty, pos=0.85] (rend) {}
            (learner);
        \node [data, above left=1.2cm and 1.0cm of learner] (dsl) {DSL $\dsl$};
        \draw (dsl) to
            node[empty, midway] (lend) {}
            (learner);
        \draw (analyzer.south east) to["question $q$" edgelabel, bend left=10] (user);
        \draw (user) to["response $r$" {edgelabel, below, pos=0.37}, bend left=10]
            node[empty, pos=0.88] (rstart) {}
            ($(analyzer.south east) - (0.4,0)$);
        \draw[pass] (rstart) to["refined spec $\spec'$"' edgelabel, bend right=20] (rend);
        \draw[pass] (lstart) to["refined DSL $\dsl'$"' edgelabel, bend right=30] (lend);
    \end{tikzpicture}
    % \uwsinglespace
    \caption{Learner-user communication in interactive PBE.
        At each iteration, the hypothesizer takes from the learner the VSA $\vsa$ of the programs consistent with the
        current spec $\spec$, and performs two automatic optimizations:
        (i) it converts the VSA into a refined DSL $\dsl'$ to be used at the next iteration instead of the original DSL
        $\dsl$, and
        (ii) it constructs the most effective clarifying question $q$ to ask the user about the ambiguous candidates in
        $\vsa$, and converts the user's response $r$ into the corresponding refined spec $\spec'$.}
    \label{fig:feedback:hypothesizer}
\end{figure}


\paragraph{Correctness-burden balance}
The crucial part of the hypothesizer's job is evaluation of disambiguation effectiveness of every potential question.
Our prototype implementation of \ConversationalClarification in \FlashProg always requested its clarification on the
first available discrepancy.
However, this is neither the fastest nor the most user-friendly experience:
\begin{itemize}
    \item On one hand, a system might ask for an additional example every time there exists an alternative candidate
        program that disagrees with the current most likely candidate on some row.
        In this case, the learning session is guaranteed to converge to the desired program if one exists, typically
        within 10-20 iterations.

        Despite the attractive guarantee, this approach is undesirable because a good ranking function often picks the
        intended program after 1-5 examples.
        In such a case, all subsequent iterations only eliminate implausible candidates without actually changing the
        result.
        The user spends a lot of cognitive effort answering superfluous questions.

    \item On another hand, the system might stop asking disambiguating questions too early.
        In this case, the learned program may be incorrect but the user will only notice it if she discovers a
        discrepancy manually.
\end{itemize}
In machine learning terminology, a superfluous disambiguating question is a \emph{false positive} (it is generated but
not strictly required to learn the correct program), and a missing disambiguating question is a \emph{false negative}
(it is required but not generated).
Our \FlashProg study~\cite{flashprog} included only the most conservative disambiguating algorithm, which always
converged to zero ambiguities.
As a result, it always generated false positive questions.

To minimize false positives, a disambiguating algorithm must intelligently pick the discrepancies presented to the user.
Well-chosen discrepancies will eliminate implausible candidate programs from consideration faster.
To minimize false negatives, the algorithm must intelligently estimate the likelihood of the currently chosen candidate
program.
A good estimation should ensure that any correct (or at least plausible) program is ranked higher than all incorrect
(implausible) programs.

\subsection{Problem Definition}
Let $\dsl$ be a DSL.
Let $\Psi$ be a set of top-level \emph{constraint types} supported by the synthesizer and witness functions for $\dsl$.
For each constraint type $\constraint \in \Psi$ we associate a \emph{descriptive question} $q$ such that a
response $r$ for this question directly constitutes an instance of~$\constraint$.
We denote such a constraint as $\Psi(r)$.
Questions $q$ can be Boolean (usually in ternary logic) or multiple-choice.
We denote the set of possible responses for $q$ as $R(q)$.

\begin{example}
    An \emph{example constraint} ``output $= v$'' corresponds to a multiple-choice question ``Is the desired output on
    an input $\state$ equal to $v_1, v_2, \dots, \text{ or } v_k$?''
    A response to this question constitutes an example constraint ``output $= v_i$'' for the chosen $i$.

    A \emph{datatype constraint} ``$\text{output}\colon \tau$'' corresponds to a Boolean question ``Is the desired
    program a computation of type $\tau$?
    Yes (always), no (never), or unknown (maybe).''

    Questions, like constraints, can be domain-specific.
    In FlashFill, a \emph{relevance constraint} states ``an input $v_i$ must/must not appear in the program.''
    It corresponds to a Boolean question ``Should the input $v_i$ be used?
    Yes (always), no (never), or unknown (maybe).''
\end{example}

\begin{figure}[t]
    \centering
    \uwsinglespace
    \begin{algorithmic}[1]
        \NoNumber \Procedure{Disambiguate}{Candidate programs $\vsa$, current spec $\spec$}
        \addtocounter{ALG@line}{-1} \WithNumber
        \State Analyze the ambiguities in $\vsa$ w.r.t.
$\spec$.
        \Statex Let $Q$ be a set of questions that may resolve ambiguity in $\vsa$
        \Statex \Comment{Compare the disambiguation scores of all questions}
        \State $q^{*} \gets \argmax\limits_{q \in Q} \mathsf{ds}(q, \vsa, \spec) $
        \If{$\mathsf{ds}(q^{*}, \vsa, \spec) < \text{threshold } T$}
        \State \textbf{break}
        \Else
        \State Present the question $q^{*}$ to the user
        \State Let $r$ be the user's response to $q$
        \State Let $\constraint$ be the response $r$ converted into a constraint
        \State \Return $\constraint$ to the learner and invoke a new round of synthesis
        \EndIf
        \EndProcedure
    \end{algorithmic}
    \vspace{\baselineskip}
    \caption{The hypothesizer's proactive disambiguation algorithm.}
    \label{fig:feedback:algorithm}
\end{figure}

\Cref{fig:feedback:algorithm} shows the disambiguation algorithm of the hypothesizer.
Given a VSA $\vsa$ of current candidate programs and the current iteration's spec $\spec$, the job of the hypothesizer
is to analyze~$\vsa$ and pick the best question to resolve ambiguities in~$\vsa$.
If~$\vsa$ has no ambiguities, or if the hypothesizer is not confident in the effectiveness of potential questions, it
considers the current candidate program $P^{*} = \mathsf{Top}_h \left(\vsa, 1\right)$ correct and does not ask any
questions.

\subsection{Disambiguation Score}

To evaluate a question's effectiveness, the hypothesizer is parameterized with a \emph{disambiguation score} function
$\mathsf{ds}(q, \vsa, \spec)$.
Higher disambiguation scores correspond to more effective questions $q$ -- that is, constraints generated by answering
$q$ eliminate more incorrect programs from $\vsa$.
Since the hypothesizer cannot predict the user's response, $\mathsf{ds}(q, \vsa, \spec)$ must represent \emph{potential
effectiveness} of $q$ for any possible outcome.

Disambiguation score functions may be domain-specific or general-purpose.
In our evaluation, we found different functions to perform well for different DSLs.
In this section, I present two efficient \emph{general-purpose} disambiguation score function.
One of them is independent of the current iteration's spec $\spec$ but takes into account the ranking scores of
alternative candidate programs in $\vsa$.
Another one is independent of any domain-specific details of the DSL (including its ranking function), but may be harder
to compute.

\paragraph{Ranking-based disambiguation}
The \emph{ranking-based disambiguation score} function prefers a question that promotes higher-ranked programs:
\[
    \mathsf{ds_R}(q , \vsa, \spec) \bydef \min_{r \in R(q)} \max_{P \in \vsa_r} h(P)
\]
where $\vsa_r$ is a set projection $\mathsf{Filter}\left(\vsa, \Psi(r)\right)$, and $h$ is a ranking function provided
with the DSL.
In other words, $\mathsf{ds_R}(q , \vsa, \spec)$ is higher if every response for the question $q$ leads to a
higher-ranked alternative program among the candidates that are consistent with it.

This disambiguation score can be efficiently evaluated for many constraint types.
For instance, calculating $\mathsf{ds_R}(q , \vsa, \spec)$ for \emph{example constraints} amounts to clustering $\vsa$
and comparing the top-ranked programs across all clusters.
Alternatively, we can quickly compute a good approximation to $\mathsf{ds_R}(q , \vsa, \spec)$ by randomly sampling $k$
programs from the VSA and considering only their outputs.

\paragraph{Entropy-based disambiguation}
The \emph{entropy-based disambiguation score} prefers a question that resolves more uncertainty in the VSA:
\[
    \mathsf{ds_E}(q , \vsa) = E \left\{ \vsa_r \,\middle|\, r \in R(q) \right\}
\]
where each $\vsa_r$ is a \emph{projection} of $\vsa$ onto the corresponding constraint for the response $r$:
\[
    \vsa_r = \mathsf{Filter}\left(\vsa,\, \Psi(r)\right)
\]
and $E(\vsa_1, \dots, \vsa_k)$ is the \emph{entropy of sizes} of its parameters:
\[
    E(\vsa_1, \dots, \vsa_k) = - \sum_{i=1}^k \frac{|\vsa_i|}{Z} \log \frac{|\vsa_i|}{Z}, \qquad
    Z = |\vsa_1| + \dots + |\vsa_k|
\]
In other words, $\mathsf{ds_E}(q , \vsa)$ is higher if possible responses for $q$ induce a high-entropy partitioning of
$\vsa$ (i.e., they split $\vsa$ into fairly even partitions).

\begin{example}
    If $q$ is a question for an \emph{example constraint}, its set of responses $R(q)$ includes all possible outputs
    $r_1, \dots, r_k$ of the programs currently in $\vsa$.
    The corresponding constraint $\Psi(r_j)$ for each response is the example constraint ``output $= r_j$''.
    Thus, the projections $\mathsf{Filter}\left(\vsa, \Psi(r_1)\right), \dots, \mathsf{Filter}\left(\vsa,
    \Psi(r_k)\right)$ are simply clusters in the partitioning $\clustering$ of $\vsa$ w.r.t. the current input $\state$.
    Each of the clusters contains only the programs that evaluate to the same output $r_j$.

    Hence, $\mathsf{ds_E}(q , \vsa)$ in this case is equal to the entropy of the clustering $\clustering$.
    Given a VSA $\vsa$, the hypothesizer simply computes its clustering on the current input $\state$, computes the
    entropy of this clustering, and uses it to decide the potential effectiveness of asking a multiple-choice
    disambiguation question.

    For example, consider a VSA of 10 programs that produce 2 different outputs.
    $R(q)$ then has two options, one for each possible output.
    If the cluster sizes corresponding to these outputs are $\{5; 5\}$, then the entropy of this clustering is high, and
    any response to $q$ will eliminate a significant fraction of the candidate programs.
    In contrast, if the cluster sizes are $\{9; 1\}$, then the entropy of this clustering is low, and a response may or
    may not be efficient in eliminating incorrect programs.
\end{example}

\begin{example}
    For a ternary Boolean question $q$ (e.g., a datatype question) the response set $R(q) = \left\{\text{``yes''},
        \text{``no''}, \text{``unknown''}\right\}$.
    The first two responses split $\vsa$ into two disjoint partitions.
    The third response does not provide any new information, so its corresponding projection is the entire $\vsa$, which
    contributes 0 to the entropy summation.
    Therefore, $\mathsf{ds_E}(q , \vsa)$ is equal to the entropy of the two-way split of $\vsa$ into programs satisfying
    $q$ and all others.
\end{example}

As compared to ranking-based disambiguation score, the entropy-based score is independent of a ranking function and
therefore DSL-agnostic.
However, it also has its drawbacks, such as reliance on the clustering operation $\clustering$, which may be expensive
for large VSAs.
% We evaluate the effectiveness of both disambiguation scores in \Cref{sec:evaluation}.

\subsection{Evaluation}

We use the PROSE-based implementation of FlashFill to evaluate feedback-based synthesis on a set of 457 text
transformation tasks.
It uses \emph{example} constraint questions.
In order to efficiently generate these questions, we sample $2000$ programs from the VSA and cluster on those, assigning
the disambiguation score $\mathsf{ds_R}$ as described above.
We chose $2000$ to achieve a good balance between performance and having a high probability of including
at least one program from every large cluster.

In the \emph{baseline setting}, the user provides the earliest incorrect row as the next example at each iteration.
In the \emph{feedback-driven setting}, the system instead proactively asks the user disambiguating questions on selected
input rows until the disambiguation score falls below the threshold $T$.
We set $T = 0.47$ as the mean of the score distribution over our tasks.

We evaluate FlashFill's feedback on two dimensions: \emph{cognitive burden} and \emph{correctness}.
Cognitive burden is defined as the number of rows the user has to \emph{read and verify} in the process.
In the baseline setting, it is the number of examples $+$ the number of correct rows before the first discrepancy that
are verified at each iteration.
In the feedback-driven setting, it is the number of questions answered.

Correctness is a combination of \emph{false positives} and \emph{false negatives}.
False positive questions occur when FlashFill keeps asking questions after the program is already correct.
False negative questions occur when FlashFill stops before it finds a correct program.

In correctness evaluation, only 2/457 tasks completed incorrectly (i.e., with false negatives).
The majority of tasks (342/457) finish with the same number of examples, and the number of false positives in the rest
never exceeds 4 (specifically, 90 tasks with 1 false positive, 22 with 2, and 1 task with 4 false positives).

\begin{table}[t]
    \centering
    \begin{tabular}{llllll}
        \toprule
        & \multicolumn{5}{c}{\textbf{Feedback-driven}} \\
        \cmidrule{2-6}
        \textbf{Baseline} & \textbf{1} & \textbf{2} & \textbf{3} & \textbf{4} & \textbf{5} \\
        \midrule
        \textbf{1} & 241 & 50 & 16 &   & 1 \\
        \textbf{2} & 75 & 30 & 4 &   &   \\
        \textbf{3} & 20 & 7 & 2 &   &   \\
        \textbf{4} &   & 5 & 3 &   &   \\
        \textbf{5} &   & 1 &   &   &   \\
        \textbf{6} & 1 & 1 &   &   &   \\
        \bottomrule
    \end{tabular}
    \uwsinglespace
    \caption{Number of rows inspected in the baseline and the feedback-driven settings for the evaluation of
        a feedback-driven FlashFill interaction with a ranking-based disambiguation score.
        The data is presented as a histogram: for example, there were \textbf{20/457} scenarios such that the baseline
        setting required \textbf{3} iterations to complete the task, and the feedback-based setting required \textbf{1}
        iteration.
        Empty cells represent the value of $0$.}
    \label{tbl:feedback:ff:iterations}
\end{table}

\Cref{tbl:feedback:ff:iterations} compares the cognitive burden of both settings.
It shows the histogram distribution of our tasks for each pair of verified row counts in baseline and feedback-driven
settings.
The baseline setting often requires the user to inspect more rows (that is, the numbers \emph{below} the diagonal in
\Cref{tbl:feedback:ff:iterations} are larger than the numbers above it).
\todoinline{Evaluation of the entropy-based disambiguation.}
