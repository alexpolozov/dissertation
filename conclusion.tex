\chapter{Conclusions and Future Work}
\label{ch:conclusion}

This work presents a new universal framework for programming by examples (PBE) that is parameterized by a
rich domain-specific language (DSL) and uses deductive search to efficiently construct a set of programs consistent with
a given inductive specification.
In the thesis statement, I set out to show that this proposed methodology generalizes most of the prior approaches to
PBE.
In \Cref{ch:background,ch:vsa,ch:prose}, I introduce, explore, and apply this methodology, building a single
unification to a disparate family of state-of-the-art techniques in the field of PBE.
Moreover, as \Cref{ch:prose} demonstrates, these technologies after a reimplementation on top of our PROSE framework
become more maintainable, extensible, and amenable to mass-market deployment.
The fact that a PBE-based technology can now be implemented on top of one common scaffolding significantly lowers the
entry barrier to developing a new such technology.
As a result, PROSE allowed us to create a family of new PBE-based tools in various data wrangling and manipulation
domains, which by now have been deployed in numerous industrial applications and re-used by external academic
collaborators.

As we discovered in \Cref{ch:interactive}, after a PBE-based technology is made available as part of a mass-market
application, the demands of superb user experience give birth to multiple unique research challenges, which do not arise
as prominently at a smaller scale.
The main such challenge is disambiguation of user intent.
While omnipresent at any scale, it is particularly challenging when the underlying DSL is a large real-life data
wrangling language and the task requires transforming or extracting a gigabyte-sized dataset.
In such a setting, the user cannot examine the entire dataset or all the outputs of the currently learned candidate
program.
As a result, intent disambiguation, previously mostly reactive, has to become proactive.
At each iteration of an interactive synthesis session the system must seek the user's feedback about the currently
learned program, suggest additional clarifying examples, and be transparent about its current understanding of the
intent.
As our experiments show, such user interaction models significantly improve the user's confidence and trust in the
system, as well as decrease the number of errors in the completed tasks.

\Cref{ch:interactive} presents our initial exploration of the interactive synthesis space.
I aim to improve the interaction process in two ways:
(a) reducing the number of refining iterations (thereby converging toward the right program faster and improving the
user's confidence in the results), and
(b) incrementally leveraging the results of past learning iterations to boost the performance of the synthesis engine
(thereby shortening the entire session).
I present our domain-agnostic implementation of feedback-based and incremental synthesis on top of the PROSE framework,
which makes them immediately available to any derived PBE technology.
The initial results show significant improvements in both the number of synthesis iterations and the learning time of a
single iteration on a diverse collection of real-life scenarios.

\section{Future Horizons}

The PROSE framework makes a significant step toward democratizing program synthesis and making it a true industrial
commodity, available to any domain expert without a required background in formal methods.
However, this step does not yet bring us all the way to this goal.
Developing a PBE technology on top of PROSE still requires non-trivial insight about both the underlying DSL and the
synthesis process.
Specifically,
\begin{itemize}[nosep]
    \item the designer must manually explore a large dataset of practical tasks to envision a concise but expressible
        enough DSL to cover all of them,
    \item the designer must implement witness functions for any non-standard operators in this DSL (which may be
        non-trivial for the operators that are not easily invertible),
    \item the designer must specify a ranking function to disambiguate among the programs in the DSL (which is tedious
        and typically requires taking into account numerous corner cases), and
    \item the designer must maintain all three aforementioned components manually, extending them to cover new tasks in
        the domain, to improve the synthesizer's performance, or to simplify the interaction process.
\end{itemize}
To truly democratize program synthesis, we should address these barriers, either via intelligent automation, or via
computer-aided tooling.
Below, I identify several specific problems in this space that we are currently investigating.

\paragraph{Learning to rank}
The most time-consuming part of developing a PROSE-based PBE technology is writing a ranking
function~\cite{polozov2016program}.
Its nature (typically fuzzy, hardly interpretable, and full of corner cases) suggests that stochastic techniques such as
deep learning should be effective at automatically learning a ranking function given a dataset of completed tasks.
In \citeyear{cav:ranking}, \citet{cav:ranking} successfully demonstrated such a technique on the original implementation
of FlashFill.
We are currently incorporating this capability in the core of the PROSE framework.

Ideally, we would like to make the training procedure domain-agnostic and thus potentially applicable to any PROSE-based
DSL.
However, this would eliminate a whole class of important features for the ranking functions -- \emph{data-based
features}.
Our recent experiments show that an effective ranking function cannot be defined on a program AST alone, it must also
take into consideration the task spec provided to the synthesis engine (i.e. program inputs and
outputs)~\cite{learning2rank}.
Hence, the final architecture will necessarily include both domain-agnostic and domain-specific components, and
designing it in an accessible way is an open question.

\paragraph{Ranking the search space}
The witness functions of PROSE implicitly define a search space for the programs that are consistent with a given task
spec.
This search space is structured as a highly branching DAG.
As the branching factor of this DAG grows, so does the duration of the learning process (as we showed in
\Cref{sec:prose:discussion}).
Because deductive search must independently explore all the branches to guarantee completeness (i.e. to find a
conformant program if one exists), such branching becomes a bottleneck in the synthesis process.

While the ranking function on its own does not help to prioritize or eliminate any branches (since it only ranks
complete programs), the ranking principle may nevertheless help us.
Recently, \citet{deepcoder} showed that a deep learning system trained with randomly generated DSL programs can learn a
probability distribution over the DSL operators that improves the performance of an enumerative program synthesizer by
several orders of magnitude.
A natural extension of this methodology to deductive synthesis is learning \emph{search tactics}.
A search tactic is a probability distribution over the branches deduced by a witness function, which helps the system
prioritize those branches that are more likely to contain a relevant subexpression for the given task spec.
For example, in real-life FlashFill scenarios it is rare that two logically different subexpressions of a
\dslinline|Concat| program construct different consecutive parts of the same token in the output string (e.g. a number).
Thus, a common effective search tactic suggests prioritizing the branches that split the output string only at token
boundaries.
Although such tactics are currently specified manually by the DSL designers, they look easily learnable given our
large compiled dataset of real-life synthesis tasks.

\paragraph{Multi-modal specifications}
Input-output examples are a natural specification choice for end users, who typically do not know formal logic or even
programming.
However, in some cases (e.g. SQL or CSS synthesis) even specifying a full example may be overly cumbersome.
A better means of specification in those cases is \emph{natural language} -- either via describing the desired program
completely, or via augmenting other specifications by keyword-based queries.
Although prior work in program synthesis by natural language has required significant
engineering~\cite{nlyze}, recent advances in natural language processing (NLP), powered by deep
learning, should help overcome this barrier~\cite{quirk2015language}.

\paragraph{Mixed-initiative DSL design}
Designing the right DSL for a PBE-based application remains in some ways an art form.
On one hand, it must be expressive enough to cover the desired practical tasks, and generalize them enough to not
overfit to the specific instances considered by the designer.
On the other hand, it must be concise and modular enough to enable efficient program synthesis via deductive search.
Currently, PROSE does not offer the designer any help in this process beyond straightforward syntax validation.
To improve this, we are considering a mixed-initiative tool that would analyze the DSL currently specified by
the domain expert, and evaluate it w.r.t. the specified design goals (such as the desired task space coverage).
