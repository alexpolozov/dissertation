\section{Intuition}
\label{sec:prose:intuition}
% \todoinline{Mark all \textbf{(a)}, \textbf{(b)}, \dots\ as subfigures with hyperlinks.}

\begin{figure}[p!]
    \begin{fullpage}
        \centering
        \uwsinglespace
        \small
        \hfill\textbf{(a)}
        \vspace{-0.8\baselineskip}
        \begin{algorithmic}
            \Function{GenerateSubstring}{$\sigma$: Input state, $s$: String}
            \State $result \gets \emptyset$
            \ForAll{\hlc{ $(i, k)$ s.t. $s$ is substring of $\sigma(v_i)$ at position $k$ }}
            \State $Y_1 \gets$ \Call{GeneratePosition}{\hlc{$\sigma(v_i), k$}}
            \State $Y_1 \gets$ \Call{GeneratePosition}{\hlc{$\sigma(v_i), k + \mathsf{Length}(s)$}}
            \State $result \gets result \cup \left\{\mathtt{SubStr}(v_i, Y_1, Y_2)\right\}$
            \EndFor
            \State \Return $result$
            \EndFunction

            \Function{GeneratePosition}{$s$: String, $k$: int}
            \State $result \gets \left\{\mathtt{CPos}(\hlc{$k$}), \mathtt{CPos}(\hlc{$-(\mathsf{Length}(s)-k)$})\right\}$
            \ForAll{\hlc{$r_1 = \mathtt{TokenSeq}(T_1, \dots, T_n)$ matching $s[k_1 : k-1]$ for some $k_1$}}
            \ForAll{\hlc{$r_2 = \mathtt{TokenSeq}(T'_1, \dots, T'_m)$ matching $s[k : k_2]$ for some $k_2$}}
            \State \hlc{$r_{12} \gets \mathtt{TokenSeq}(T_1, \dots, T_n, T'_1, \dots, T'_m)$}
            \State \hlc{Let $c$ be s.t. $s[k_1 : k_2]$ is the $c^{\text{th}}$ match for $r_{12}$ in $s$}
            \State \hlc{Let $c'$ be the total number of matches for $r_{12}$ in $s$}
            \State $\tilde{r_1} \gets$ \Call{GenerateRegex}{$r_1, s$}
            \State $\tilde{r_2} \gets$ \Call{GenerateRegex}{$r_2, s$}
            \State $result \gets result \cup \left\{\mathtt{Pos}(\tilde{r_1}, \tilde{r_2}, \left\{\hlc{$c$}, \hlc{$-(c'-c+1)$}\right\})\right\}$
            \EndFor
            \EndFor
            \State \Return $result$
            \EndFunction
        \end{algorithmic}
        \vspace{5pt}
        \hrule
        \vspace{3pt}\hfill\textbf{(b)}
        \vspace{-0.9\baselineskip}
        \begin{algorithmic}
            \Function{Map.Learn}{Examples $\spec$: $\mathsf{Dict}\langle \mathsf{State}, \mathsf{List}\langle T\rangle\rangle$}
            \State Let $\spec$ be $\left\{\state_1 \mapsto Y_1, \dots, \state_m \mapsto Y_m\right\}$
            \For{$j \gets 1\dots m$}
            \State Witness subsequence \hlc{$Z_j \gets \mathsf{Map.Decompose}(\state_j, Y_j)$}
            \EndFor
            \State $\spec_1 \gets$ \hlc{$\bigl\{\state[Z_j[i] / x] \mapsto Y_j[i] \mid i = 0..|Z_j|-1,\, j = 1..m\bigr\}$}
            \State $\vsa_1 \gets F.\mathsf{Learn}(\spec_1)$
            \State $\spec_2 \gets$ \hlc{$\left\{\state_j \mapsto Z_j \mid j = 1..m\right\}$}
            \State $\vsa_2 \gets S.\mathsf{Learn}(\spec_2)$
            \State \Return $\mathsf{Map}(\vsa_1, \vsa_2)$
            \EndFunction
            \\
            \Function{Filter.Learn}{Examples $\spec$: $\mathsf{Dict}\langle \mathsf{State}, \mathsf{List}\langle T\rangle\rangle$}
            \State $\vsa_1 \gets S.\mathsf{Learn}(\hlc{$\spec$})$
            \State $\spec' \gets$ \hlc{$\bigl\{ \state[Y[i]/x] \mapsto \mathsf{true} \mid (\state, Y) \in \spec,\, i = 0..|Y|-1\bigr\}$}
            \State $\vsa_2 \gets F.\mathsf{Learn}(\spec')$
            \State \Return $\mathsf{Filter}(\vsa_2, \vsa_1)$
            \EndFunction
        \end{algorithmic}
        \vspace{3pt}
        \caption{\textbf{(a)} FlashFill synthesis algorithm for learning substring expressions \cite[Figure 7]{flashfill};
            \textbf{(b)} FlashExtract synthesis algorithm for learning $\mathsf{Map}$ and $\mathsf{Filter}$ sequence
            expressions \cite[Figure 6]{flashextract}.
            Highlighted portions correspond to domain-specific computations, which deduce I/O examples for propagation in
            the DSL by ``inverting'' the semantics of the corresponding DSL operator.
        Non-highlighted portions correspond to the search organization, isomorphic between FlashFill and FlashExtract.}
        \label{fig:prose:prior}
    \end{fullpage}
\end{figure}

The main observation behind PROSE is that most of prior synthesis algorithms in PBE can be decomposed into
\begin{enumerate*}[label=\textbf{(\alph*)}]
    \item domain-specific insights for refining a spec for the operator into the specs for its
        parameters, and
    \item a common deductive search algorithm, which leverages these insights for iterative reduction of the synthesis
        problem
\end{enumerate*}.

For instance, \Cref{fig:prose:prior} shows a portion of the synthesis algorithms for FlashFill \cite[Figure
7]{flashfill} and FlashExtract~\cite[Figure 6]{flashextract}.
Both algorithms use a divide-and-conquer approach, reducing a synthesis problem for an expression into smaller synthesis
problems for its subexpressions.
They alternate between three phases, implicitly hidden in the ``fused'' original presentation:
\begin{enumerate}[nosep]
    \item Given some examples for a nonterminal $N$, invoke synthesis on all RHS rules of $N$, and unite the results.
    \item \emph{Deduce} examples that should be propagated to some argument $N_j$ of a current rule $N := F(N_1, \dots,
        N_k)$.
    \item Invoke learning recursively on the deduced examples, and proceed to phase 2 for the subsequent arguments.
\end{enumerate}
Note that phases 1 and 3 do not exploit the \emph{semantics} of the DSL, they only process its \emph{structure}.
In contrast, phase 2 uses domain-specific knowledge to deduce propagated examples for $N_j$ from the examples for $N$.
In \Cref{fig:prose:prior}, we highlight parts of the code that implement phase 2.

It is important that phases 2 and 3 interleave as nested loops for each argument $N_j$ of the currently synthesized
rule, because the examples deduced for an argument $N_{j}$ may depend on the possible value of the
previously processed argument $N_{j-1}$.
For instance, the example positions $k$, deduced for the nonterminal $p$ in \textsc{GenerateSubstring}, depend on the
currently selected possible input string $v_i$.

In FlashExtract, list processing operators $\mathsf{Map}(F, S)$ and $\mathsf{Filter}(F, S)$ are learned generically from
a similar \emph{subsequence spec} $\spec$ of type $\{ \state \tospec \textbf{?} \sqsupseteq
\mathsf{List}\langle T\rangle \}$.
Their learning algorithms are also shown in \Cref{fig:prose:prior}.
Similarly, they are organized in the ``divide-and-conquer'' manner, wherein the given spec is first transformed
into the corresponding parameter specs, and then the synthesis on them is invoked recursively.
FlashExtract provides generic synthesis for five such list\hyp{}processing operators independently of their DSL
instantiations.
However, the domain-specific insight required to build a spec for the $\mathsf{Map}$'s $F$ parameter depends on
the specific DSL instantiation of this $\mathsf{Map}$ (such as $\mathsf{LinesMap}$ or $\mathsf{StartSeqMap}$ in
$\fedsl$), and cannot be implemented in a domain-independent manner.
This insight was originally called a \emph{witness function} -- it \emph{witnessed} sequence elements passed to $F$.
We notice, however, that FlashExtract witness functions are just instances of domain-specific knowledge that
appears in phase~2 of the common meta-algorithm for inductive synthesis.

\subsection{Derivation}
Consider the operator $\mathsf{LinesMap}(F, L)$ in $\fedsl$ (see \Cref{fig:dsl:flashfill}).
It takes as input a \mbox{(sub-)sequence} of lines $L$ in the document, and applies a function $F$ on each line in $L$,
which selects some substring within a given line.
Thus, the simplest type signature of $\mathsf{LinesMap}$ is \mbox{$\left((\mathsf{String} \!\to\!
\mathsf{String}),\, \mathsf{List}\langle\mathsf{String}\rangle\right) \to \mathsf{List}\langle \mathsf{String}
    \rangle$}.

The simplest form of \Cref{problem:syn:general} for $\mathsf{LinesMap}$ is \emph{``Learn all programs of kind
    $\mathsf{LinesMap}(F, L)$ that satisfy a given I/O example $\spec = \state \rightsquigarrow v$''}.
Here $v$ is an output value, i.e. a list of extracted strings.
To solve this problem, we need to compute the following set of expressions:
\begin{equation}
    \left\{\left(F, L\right) \mid \mathsf{LinesMap}(F, L) \models \spec \right\}
    \label{eqn:derivation:problem}
\end{equation}
Applying the principle of divide-and-conquer, in order to solve \eqref{eqn:derivation:problem}, we can compute an
\emph{inverse semantics} of $\mathsf{LinesMap}$:
\begin{equation}
    \mathsf{LinesMap}^{\inv}(v) \bydef \left\{(f, \ell) \mid \mathsf{LinesMap}(f, \ell) = v\right\}
    \label{eqn:derivation:inverse}
\end{equation}
Then, we can recursively learn all programs $F$ and $L$ that evaluate to $f$ and $\ell$ respectively on the input state
$\state$.

\paragraph{Finitization}
Computation of $\mathsf{LinesMap}^{\inv}(v)$ is challenging.
One possible approach could be to enumerate all argument values $f \in \codomain F$, $\ell \in \codomain L$, retaining
only those for which $\mathsf{LinesMap}(f, \ell) = v$.
However, both $\codomain L$ (all string lists) and $\codomain F$ (all $\mathsf{String} \to \mathsf{String}$ functions)
are infinite.\footnote{%
    Here $\codomain F$ denotes the \emph{codomain} of an operator $F$, i.e. the set of its possible outputs.
}

In order to finitize the value space, we apply domain-specific knowledge about the behavior of $\mathsf{LinesMap}$ in
$\fedsl$.
Namely, we use that $L$ must evaluate to a sequence of lines in the input document, and~$F$ must evaluate to a function
that selects a subregion in a given line.
Thus, the ``strongly-typed'' signature of $\mathsf{LinesMap}$ is actually $((\mathsf{Line} \!\to\!
\mathsf{StringRegion}),\,
\mathsf{List}\langle \mathsf{Line}\rangle) \to \mathsf{List}\langle \mathsf{StringRegion}\rangle$
where $\mathsf{StringRegion}$ is a domain-specific type for ``a region within a given document''.
This is a \emph{dependent type}, parameterized by the state $\state$ that contains our input document.
Thus, the inverse semantics $\mathsf{LinesMap}^{\inv}$ is also implicitly parameterized with $\sigma$:
\begin{equation*}
    \mathsf{LinesMap}^{\inv}(\state \rightsquigarrow v) =
    \left\{(f, \ell) \mid
        f \text{ and } \ell \text{ can be obtained from } \state \ \wedge\ \mathsf{LinesMap}(f, \ell) = v
    \right\}
\end{equation*}

Our implementation of $\mathsf{LinesMap}^{\inv}(\state \rightsquigarrow v)$ now enumerates over all line sequences
$\ell$ and substring functions $f$ \emph{given an input document}.
Both codomains are finite.
Note that we took into account both input and output in the spec to produce a constructive synthesis procedure.

\paragraph{Decomposition}
The synthesis procedure above is finite, but not necessarily efficient.
For most values of~$\ell$, there may be no satisfying program $L$ in the DSL to produce it, and therefore even
computing matching values of $f$ for it is redundant.
For instance, let $v = [r_1, r_2]$.
In this case $\ell$ must be the list of two document lines $[l_1, l_2]$ containing $r_1$ and $r_2$, respectively; any
other value for $\ell$ cannot yield $v$ as an output of $\mathsf{LinesMap}(f, \ell)$ regardless of $f$.
In general, \emph{for many domain-specific operators $F(X, Y)$ computing all viable
arguments $(x, y)$ is much slower than computing matching $y$s only for the realizable values of $X$}.

This observation leads to another key idea in PROSE: \emph{decomposition} of inverse semantics.
Namely, for our $\mathsf{LinesMap}$ example, instead of computing $\mathsf{LinesMap}^{\inv}(\state \rightsquigarrow v)$
explicitly, we ask two simpler questions separately:
\begin{enumerate}[nosep]
    \item If $\semantics{\mathsf{LinesMap}(F, L)}{\state} = v$, what could be the possible output of $L$?
    \item If $\semantics{\mathsf{LinesMap}(F, L)}{\state} = v$ and we know that $\semantics{L}{\state} = \ell$, what
        could be the possible output of $F$?
\end{enumerate}
The answers to these questions define, respectively, \emph{partial} and \emph{conditional} inverse semantics of
$\mathsf{LinesMap}$ with respect to its individual parameters:
\begingroup\allowdisplaybreaks
\begin{align}
    \mathsf{LinesMap}^{\inv}_L(v) \ &\supset\ \left\{ \ell \mid \exists f\colon \mathsf{LinesMap}(f, \ell) = v \right\}
    \label{eqn:derivation:partial} \\
    \mathsf{LinesMap}^{\inv}_F (v \assuming L = \ell) \ &=\ \left\{ f \mid \mathsf{LinesMap}(f, \ell) = v \right\}
    \label{eqn:derivation:conditional}
\end{align}
\endgroup

The key insight of this technique is that, by the principle of \emph{skolemization}~\cite{modeltheory}, inverse
semantics
$\mathsf{LinesMap}^{\inv}(v)$ can be expressed as a cross-product computation of parameter inverses
$\mathsf{LinesMap}^{\inv}_L(v)$ and $\mathsf{LinesMap}^{\inv}_F(v \assuming L = \ell)$ for all possible $\ell \in
\mathsf{LinesMap}^{\inv}_L(v)$:
\begin{gather}
    \begin{aligned}
        \mathsf{LinesMap}^{\inv}(v) = &\left\{(f, \ell) \mid \mathsf{LinesMap}(f, \ell) = v\right\} \\
        = &\left\{(f, \ell) \mid \ell \in \mathsf{LinesMap}^{\inv}(v), f \in \mathsf{LinesMap}^{\inv}_F(v \assuming L = \ell)\right\}
    \end{aligned}
    \label{eqn:derivation:united}
    \raisetag{2\baselineskip}
\end{gather}
Such a decomposition naturally leads to an efficient synthesis procedure for solving the problem
\mbox{$\mathsf{Learn}(\mathsf{LinesMap}(F, L),\ \state \rightsquigarrow v)$}, based on a divide-and-conquer
algorithmic paradigm:
\begin{enumerate}[nosep]
    \item Enumerate  $\ell \in \mathsf{LinesMap}_L^\inv (v)$.
    \item For each $\ell$ recursively find a set of programs $\vsa_{\ell}$ rooted at $L$ such that
        $\semantics{L}{\state} = \ell$.
    \item Enumerate  $f \in \mathsf{LinesMap}_F^\inv (v \assuming L = \ell)$ for all those $\ell$ for which the program
        set $\vsa_{\ell}$ is non-empty.
    \item For each $f$ recursively find a set of programs $\vsa_{\ell,f}$ rooted at $F$ such that $\semantics{F}{\state}
        = f$.
    \item Now, for any such combination of parameter programs $L \in \vsa_{\ell}$, $F \in \vsa_{\ell,f}$ we guarantee
        \mbox{$\mathsf{LinesMap}(F, L) \models \spec$} by construction.
\end{enumerate}

\paragraph{Generalization}
In practice, \Cref{problem:syn:general} for $\mathsf{LinesMap}$ is rarely solved for example-based specs: as
discussed in \Cref{sec:background:spec}, providing the entire output $v$ of regions selected by $\mathsf{LinesMap}$ is
impractical for end users.
Thus, instead of concrete outputs the procedure above should manipulate \emph{specs with properties
    of concrete outputs} (e.g.  a prefix of the output list $v$ instead of entire $v$).
A corresponding generalization of ``inverse semantics of $\mathsf{LinesMap}$'' is a function that deduces a
spec for $L$ given a spec $\spec$ on $\mathsf{LinesMap}(F,\,L)$ (or for $F$, under the assumption of fixed $L$).
We call such a generalization a \emph{witness function}.

In our $\mathsf{LinesMap}$ example, a user might provide a \emph{prefix spec} \mbox{$\spec = \{ \state
\tospec [r_1, r_2, \dots] \}$} for the output list of regions.
The witness functions for such a spec arise naturally:
\begin{itemize}
    \item A witness function for $L$ in this case follows the same observation above: the output list of $L$ should
        begin with two lines containing $r_1$ and $r_2$.
        This is also a prefix specification.
    \item A witness function for $F$ (conditional on $\ell$, the output value of $L$) is universal for all
        $\mathsf{Map}$s: it requires the output of $F$ to be a function that maps the $i^{\text{th}}$ element of $\ell$
        into the $i^{\text{th}}$ element of $[r_1, r_2]$.
        Such modularity of synthesis (a common witness function for any domain-specific $\mathsf{Map}$ operator) is
        another advantage of decomposing inverse semantics into partial and conditional witness functions.
\end{itemize}

