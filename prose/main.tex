\chapter{The \PROSE Framework}
\label{ch:prose}

In this section, I describe the methodology of program synthesis the \PROSE framework: \emph{deductive synthesis
driven by domain-specific witness functions}.
I first introduce the notion of witness functions informally in \Cref{sec:prose:intuition}, deriving it from various
instances of prior work in PBE (such as FlashFill and FlashExtract).
Then, in \Cref{sec:prose:wf}, I give a formal definition of witness functions, as well as a comprehensive review of
their usage in PROSE-based PBE technologies.
\Cref{sec:prose:algorithm} presents the main synthesis algorithm of PROSE, \emph{deductive search}.\footnote{Also known
    in the literature as \emph{divide-and-conquer search}, \emph{top-down search}, and \emph{backpropagation-based
    search} (not to be confused with the backpropagation algorithm for training neural
    networks~\cite{chauvin1995backpropagation}, although this name was inspired by similarities between the two
techniques).}
In \Cref{sec:prose:evaluation}, I show its evaluation on $12$+ real-life case studies -- PBE technologies developed on top
of the framework.
Finally, \Cref{sec:prose:discussion} discusses various limitations and extensions to the algorithm.

\section{Intuition}
\label{sec:prose:intuition}
\todoinline{Wrap all big figures in \texttt{\textbackslash begin\{fullpage\}}.}
\todoinline{Mark all \textbf{(a)}, \textbf{(b)}, \dots\ as subfigures with hyperlinks.}

\begin{figure*}[t]
    \centering
    \uwsinglespace
    \small
    \hfill\textbf{(a)}
    \vspace{-0.8\baselineskip}
    \begin{algorithmic}
        \Function{GenerateSubstring}{$\sigma$: Input state, $s$: String}
            \State $result \gets \emptyset$
            \ForAll{\hlc{ $(i, k)$ s.t. $s$ is substring of $\sigma(v_i)$ at position $k$ }}
            \State $Y_1 \gets$ \Call{GeneratePosition}{\hlc{$\sigma(v_i), k$}}
                \State $Y_1 \gets$ \Call{GeneratePosition}{\hlc{$\sigma(v_i), k + \mathsf{Length}(s)$}}
                \State $result \gets result \cup \left\{\mathtt{SubStr}(v_i, Y_1, Y_2)\right\}$
            \EndFor
            \State \Return $result$
        \EndFunction

        \Function{GeneratePosition}{$s$: String, $k$: int}
            \State $result \gets \left\{\mathtt{CPos}(\hlc{$k$}), \mathtt{CPos}(\hlc{$-(\mathsf{Length}(s)-k)$})\right\}$
            \ForAll{\hlc{$r_1 = \mathtt{TokenSeq}(T_1, \dots, T_n)$ matching $s[k_1 : k-1]$ for some $k_1$}}
                \ForAll{\hlc{$r_2 = \mathtt{TokenSeq}(T'_1, \dots, T'_m)$ matching $s[k : k_2]$ for some $k_2$}}
                \State \hlc{$r_{12} \gets \mathtt{TokenSeq}(T_1, \dots, T_n, T'_1, \dots, T'_m)$}
                \State \hlc{Let $c$ be s.t. $s[k_1 : k_2]$ is the $c^{\text{th}}$ match for $r_{12}$ in $s$}
                    \State \hlc{Let $c'$ be the total number of matches for $r_{12}$ in $s$}
                    \State $\tilde{r_1} \gets$ \Call{GenerateRegex}{$r_1, s$}
                    \State $\tilde{r_2} \gets$ \Call{GenerateRegex}{$r_2, s$}
                    \State $result \gets result \cup \left\{\mathtt{Pos}(\tilde{r_1}, \tilde{r_2}, \left\{\hlc{$c$}, \hlc{$-(c'-c+1)$}\right\})\right\}$
                \EndFor
            \EndFor
            \State \Return $result$
        \EndFunction
    \end{algorithmic}
    \vspace{5pt}
    \hrule
    \vspace{3pt}\hfill\textbf{(b)}
    \vspace{-0.9\baselineskip}
    \begin{algorithmic}
        \Function{Map.Learn}{Examples $\spec$: $\mathsf{Dict}\langle \mathsf{State}, \mathsf{List}\langle T\rangle\rangle$}
            \State Let $\spec$ be $\left\{\state_1 \mapsto Y_1, \dots, \state_m \mapsto Y_m\right\}$
            \For{$j \gets 1\dots m$}
            \State Witness subsequence \hlc{$Z_j \gets \mathsf{Map.Decompose}(\state_j, Y_j)$}
            \EndFor
            \State $\spec_1 \gets$ \hlc{$\bigl\{\state[Z_j[i] / x] \mapsto Y_j[i] \mid i = 0..|Z_j|-1,\, j = 1..m\bigr\}$}
            \State $\vsa_1 \gets F.\mathsf{Learn}(\spec_1)$
            \State $\spec_2 \gets$ \hlc{$\left\{\state_j \mapsto Z_j \mid j = 1..m\right\}$}
            \State $\vsa_2 \gets S.\mathsf{Learn}(\spec_2)$
            \State \Return $\mathsf{Map}(\vsa_1, \vsa_2)$
        \EndFunction
        \\
        \Function{Filter.Learn}{Examples $\spec$: $\mathsf{Dict}\langle \mathsf{State}, \mathsf{List}\langle T\rangle\rangle$}
            \State $\vsa_1 \gets S.\mathsf{Learn}(\hlc{$\spec$})$
            \State $\spec' \gets$ \hlc{$\bigl\{ \state[Y[i]/x] \mapsto \mathsf{true} \mid (\state, Y) \in \spec,\, i = 0..|Y|-1\bigr\}$}
            \State $\vsa_2 \gets F.\mathsf{Learn}(\spec')$
            \State \Return $\mathsf{Filter}(\vsa_2, \vsa_1)$
        \EndFunction
    \end{algorithmic}
    \vspace{3pt}
    \caption{\textbf{(a)} FlashFill synthesis algorithm for learning substring expressions \cite[Figure 7]{flashfill};
        \textbf{(b)} FlashExtract synthesis algorithm for learning $\mathsf{Map}$ and $\mathsf{Filter}$ sequence
        expressions \cite[Figure 6]{flashextract}.
        Highlighted portions correspond to domain-specific computations, which deduce I/O examples for propagation in
        the DSL by ``inverting'' the semantics of the corresponding DSL operator.
        Non-highlighted portions correspond to the search organization, isomorphic between FlashFill and FlashExtract.}
    \label{fig:prose:prior}
\end{figure*}

\section{Witness Functions}
\label{sec:prose:wf}
As described above, a \emph{witness function} is a generalization of inverse semantics for an operator.
In other words, it is a \emph{problem reduction logic}, which deduces a necessary (or sufficient) spec on the operator's
parameters given a desired spec on the operator's output.
In this section, I define witness functions formally and give various examples of their usages in the existing PBE
technologies.

\begin{defn}[Witness function]
    Let $F\left(N_1,\dots,N_k\right)$ be an operator in a DSL $\dsl$.
    A \emph{witness function} of $F$ for $N_j$ is a function $\omega_{j}\left(\spec\right)$ that deduces a
    \emph{necessary} spec $\spec_j$ on $N_j$ given a spec $\spec$ on $F\left(N_1, \dots, N_k\right)$.
    Formally, $\omega_j\left(\spec\right) = \spec_j$ iff the following implication holds:\footnote{All free variables
        are universally quantified unless otherwise specified.}
    \begin{equation*}
        F\left(N_1, \dots, N_k\right) \models \spec \qquad\Longrightarrow\qquad N_j \models \spec_j.
    \end{equation*}
\end{defn}

\begin{defn}[Precise witness function]
    A witness function $\omega_j$ of $F\left(N_1, \dots, N_k\right)$ for $N_j$ is \emph{precise} if its deduced spec is
    \emph{necessary and sufficient}.
    Formally, $\omega_j\left(\spec\right) = \spec_j$ is precise iff
    \begin{equation*}
        N_j \models \spec_j \qquad\Longleftrightarrow\qquad
        \exists\, N_1, \dots, N_{j-1}, N_{j+1}, \dots, N_k\colon\ F\left(N_1, \dots, N_k\right) \models \spec.
    \end{equation*}
\end{defn}

\begin{defn}[Conditional witness function]
    A \emph{(precise) conditional witness function} of $F\left(N_1, \dots, N_k\right)$ for $N_j$ is a function
    $\omega_j\left(\spec \assuming N_{k_1} = v_1, \dots, N_{k_s} = v_s\right)$ that deduces a necessary (and sufficient)
    spec $\spec_j$ on $N_j$ given a spec $\spec$ on $F\left(N_1, \dots, N_k\right)$ under the assumption that a subset
    of other parameters $N_{k_1}, \dots, N_{k_s}$ of $F$ (called \emph{prerequisites}) have fixed values $v_1, \dots,
    v_k$.
    Formally, $\omega_j\left(\spec \assuming N_t = v_t\right) = \spec_j$ iff the following implication holds:
    \begin{equation*}
        F\left(N_1, \dots, N_{t-1}, v_t, N_{t+1}, \dots, N_k\right) \models \spec
        \qquad\Longrightarrow\qquad
        N_j \models \spec_j.
    \end{equation*}
\end{defn}

\begin{sidewaystable}[p!]
    \begin{fullpage}
        \centering
        % \uwsinglespace
        \small
        \begin{tabu}{lX[-1$l]X[-1$l]X[-1$l]>{\hspace*{1.2em}}X[-3$l]}
            \toprule
            \textbf{Operator} & \textbf{Input spec} & \textbf{Parameter} & \textbf{Prerequisites} & \textbf{Output
            spec} \\
            \midrule
            \dslinline|Concat($atom$,\ $transform$)| & \state \tospec w & atom & &
                \state \tospec \bigvee\limits_{j=1}^{|w|-1} w[0..j] \\
                \cmidrule{3-5}
                & & transform & atom = v & \state \tospec w[|v|..] \\
            \midrule
            \dslinline|ConstStr($s$)| & \state \tospec w & s & & \state \tospec w \\
            \midrule
            \dslinline|let\ $x\ $ = $\ \tikzmark{mark:b:Bind}$std.Kth($inputs$,\ $k$)$\tikzmark{mark:e:Bind}$| &
              \state \tospec w & binding & &
              \state \tospec \bigvee\limits_{\mathclap{j\colon w \text{ occurs in } v_j}}\ v_j \\[3pt]
              \cmidrule{3-5}
          \qquad\dslinline|in Substring($x$,\ $pp$)| & & pp & x = v & \state \tospec
              \bigvee\limits_{\mathclap{\substack{w \text{ occurs in } v \\ \text{ at position } l\ \ }}}
              \ \langle l, l + |w|\rangle \\
            \midrule
            \dslinline|AbsPos($x$,\ $k$)| & \state \tospec c & k & x = v & \state \tospec c \vee c - |v| - 1 \\
            \midrule
            \dslinline|RegexPos($x$, $\ \tikzmark{mark:b:RR}$std.Pair($r$,\ $r$)$\tikzmark{mark:e:RR}$,\ $k$)| &
              \state \tospec c & rr & x = v &
              \state \tospec \bigvee\limits_{\mathclap{\substack{
                          \langle r_1, r_2 \rangle\colon r_1 \text{ matches left of } c \\
                          \hspace*{2.5em}\wedge\ r_2 \text{ matches right of } c}}}\ \langle r_1, r_2 \rangle \\
              \cmidrule{3-5}
              & & k & \begin{aligned} &x = v,\\[0pt] &rr = \langle r_1, r_2 \rangle \end{aligned} &
                  \begin{aligned}
                      &\state \tospec j \vee j - |\vec{c}| - 1 \quad\text{ where } \\
                      &\quad \vec{c} \text{ are the matches of } \langle r_1, r_2 \rangle \text{ in } v, \\
                      &\quad j \text{ is an index of } c \text{ in } \vec{c}
                  \end{aligned} \\
            \bottomrule
        \end{tabu}
        \begin{tikzpicture}[overlay, remember picture]
            \draw[decorate, decoration={brace, mirror}, transform canvas={yshift=-0.3em}]
                (pic cs:mark:b:Bind) -- (pic cs:mark:e:Bind) node[midway, below] {$binding$};
        \end{tikzpicture}
        \begin{tikzpicture}[overlay, remember picture]
            \draw[decorate, decoration={brace, mirror}, transform canvas={yshift=-0.3em}]
                (pic cs:mark:b:RR) -- (pic cs:mark:e:RR) node[midway, below] {$rr$};
        \end{tikzpicture}
        \caption{Witness functions for various FlashFill operators.}
        \label{tbl:wfs:flashfill}
    \end{fullpage}
\end{sidewaystable}

\begin{sidewaystable}[p!]
    \begin{fullpage}
        \centering
        % \uwsinglespace
        \small
        \begin{tabu}{lX[-1$l]X[-1$l]X[-1$l]X[-3$l]}
            \toprule
            \textbf{Operator} & \textbf{Input spec} & \textbf{Parameter} & \textbf{Prerequisites} & \textbf{Output
            spec} \\
            \midrule
            \dslinline|Kth($xs$,\ $k$)| & \state \tospec w & xs & & \state \tospec [\dots, w, \dots] \\
            \cmidrule{3-5}
            & & k & xs = \vec{v} & \state \tospec \bigvee\limits_{v_j = w} j \\
            \midrule
            \dslinline|Pair(|$p_1$\dslinline|,\ |$p_2$\dslinline|)| & \state \tospec \langle v_1, v_2 \rangle & p_j & &
                \state \tospec v_j \\
            \midrule
            \dslinline|Map($F$,\ $L$)| & \state \tospec \mathbf{?} \sqsupseteq \vec{\ell} \text{ as a prefix } & F &
                L = \vec{v} & \state \tospec f \text{ s.t. } \bigwedge\limits_{i=1}^{|\vec{\ell}|} f(v_i) = \ell_i \\
            \midrule
            $\fun{x}\ b$ & \state \tospec f \text{ s.t. } f(v) = y & b & & \state[x \coloneq v] \tospec y \\
            \midrule
            \dslinline|Filter($P$,\ $L$)| & \state \tospec \mathbf{?} \sqsupseteq \vec{\ell} & L & &
                \state \tospec \mathbf{?} \sqsupseteq \vec{\ell} \\
            \cmidrule{3-5}
            & & P & L = \vec{v} & \state \tospec f \text{ s.t. }
                \bigwedge\limits_{i=1}^{|\vec{v}|} f(\ell_i) = \bigl[v_i \in \vec{\ell}\bigr] \\
            \midrule
            \dslinline|FilterInt($i$,\ $k$,\ $L$)| & \state \tospec \mathbf{?} \sqsupseteq \vec{\ell} & L & &
                \state \tospec \mathbf{?} \sqsupseteq \vec{\ell} \\
            \cmidrule{3-5}
            & & i & L = \vec{v} & \state \tospec \text{ index of } \ell_1 \text{ in } \vec{v} \\
            \cmidrule{3-5}
            & & k & L = \vec{v} &
                \begin{aligned}
                    &\state \tospec \bigvee\limits_{d\;|\; g} d \quad\text{ where } \\
                    &\quad g = \mathsf{gcd}\bigl(\Delta_1, \dots, \Delta_{|\vec{\ell}|-1}\bigr), \\
                    &\quad \Delta_i = p_{i+1} - p_i, \\
                    &\quad p_j \text{ is an index of } \ell_j \text{ in } \vec{v}
                \end{aligned} \\
            \bottomrule
        \end{tabu}
        \caption{Witness functions for some operators from the standard library of PROSE.}
        \label{tbl:wfs:prose}
    \end{fullpage}
\end{sidewaystable}



\begin{example}
    \Cref{tbl:wfs:flashfill} shows the witness functions for all $\ffdsl$ operators from \Cref{fig:dsl:flashfill}:
    \begin{itemize}[nosep]
        \item A \dslinline|Concat($atom$, $transform$)| expression returns $w$ iff $atom$ returns some prefix of $w$.
            In addition, assuming that $atom$ returns $v$, $transform$ mush return the remaining suffix of~$w$ after the
            end of $v$.
        \item A \dslinline|ConstStr($s$)| expression returns $w$ iff $s$ is equal to $w$.
        \item An expression ``\dslinline|let $x$ = std.Kth($inputs$, $k$) in $\dots$|'' returns $w$ iff $x$ is bound to
            an element of $inputs$ that has $w$ as a substring.
        \item A \dslinline|Substring($x$, $pp$)| expression returns $w$ (assuming that $x$ returns $v$) iff $pp$ returns
            a position span of any occurrence of $w$ in $v$ as a substring.
        \item An \dslinline|AbsPos($x$, $k$)| expression returns $c$ (assuming that $x$ returns $v$) iff $k$ is equal to
            either~$c$ or $c-|v|-1$ (since $k$ may represent a left or right offset depending on its sign).
        \item An expression \dslinline|RegexPos($x$, $rr$, $k$)| returns $c$ (assuming that $x$ returns $v$) iff $rr$ is
            equal to any pair of regular expressions that matches the boundaries of position $c$ in the string~$v$.
            In addition, assuming that $rr$ is equal to $\langle r_1, r_2\rangle$, $P$ returns $c$ iff $k$ is equal to
            a index of $c$ (from the left or right) among all matches of $\langle r_1, r_2\rangle$ in $v$.
    \end{itemize}
    \label{ex:wf:flashfill}
\end{example}

\begin{example}
    \Cref{tbl:wfs:prose} shows the witness functions for various operators from the standard library of PROSE that are
    used in $\ffdsl$ and $\fedsl$:
    \begin{itemize}[nosep]
        \item A \dslinline|Kth($xs$, $k$)| expression returns $w$ (given that $xs$ returns $\values$) iff $k$ is an
            index of some occurrence of $w$ in $\values$.
        \item If \dslinline|Pair($p_1$, $p_2$)| returns $\langle v_1, v_2\rangle$, then $p_j$ returns $v_j$.
            Note that this witness function is imprecise, since it restricts only a single parameter ($p_1$ or $p_2$).
        \item A \dslinline|Map($F$, $L$)| expression returns a list that starts with a sublist $\vec{\ell}$ as a prefix
            (given that $L$ returns $\vec{v}$) iff $F$ is any $\lambda$-function that maps the first $|\vec{\ell}|$
            elements of $\vec{v}$ to the corresponding elements of $\vec{\ell}$.
            The fact that $L$ returns $\vec{v}$ is usually established thanks to a corresponding domain-specific witness
            function for $L$ (defined by a DSL designer for a particular instantiation of \texttt{Map}).
        \item A \dslinline|Filter($P$, $L$)| expression returns a list that contains $\vec{\ell}$ as a sublist iff the
            result of $L$ also contains $\vec{\ell}$ as a sublist.
            In addition, assuming that $L$ returns a list $\vec{v}$, $P$ must be any $\lambda$-function that maps all
            the elements of $\vec{v}$ that are also present in $\vec{\ell}$ to \texttt{true}, and the rest of the
            elements to \texttt{false}.
        \item A \dslinline|FilterInt($i$, $k$, $L$)| expression returns a list that contains $\vec{\ell}$ as a sublist
            iff the result of $L$ also contains $\vec{\ell}$ as a sublist.
            In addition, assuming that $L$ returns a list $\vec{v}$, the initial value expression $i$ must evaluate to
            the index of $\ell_1$ in $\vec{v}$, and the step expression $k$ must evaluate to some divisor of the GCD of
            the gaps between the consecutive occurrences of the elements of $\vec{\ell}$ in $\vec{v}$.
    \end{itemize}
\end{example}

\begin{example}
    As mentioned above, a witness function for the parameter $L$ of \dslinline|Map($F$, $L$)| must be defined
    specifically for every instantiation of \texttt{Map} in a given DSL.
    For instance, the FlashExtract DSL $\fedsl$ contains three \texttt{Map} instantiations: \texttt{LinesMap},
    \texttt{StartSeqMap}, and \texttt{EndSeqMap} (as defined in \Cref{fig:dsl:flashfill}).
    Their respective witness functions $\omega_L$ for the same \emph{prefix spec} $\state \tospec \mathbf{?} \sqsupseteq
    \vec{v}$ are shown below:

    \vspace{0.5\baselineskip}
    \begin{tabu}{l>{\hspace{0.1cm}}X[$l]}
        \toprule
        \textbf{Operator} & \textbf{Output spec} \\
        \midrule
        \texttt{LinesMap} & \begin{aligned}
            &\state \tospec \mathbf{?} \sqsupseteq \bigl[ \ell_1, \dots, \ell_{|\vec{v}|} \bigr] \quad\text{ where } \\
            &\quad \ell_i \text{ is a line of the input document } \state[doc] \text{ that contains the region } v_i
        \end{aligned} \\
        \texttt{StartSeqMap} &
            \state \tospec \mathbf{?} \sqsupseteq \bigl[p_1 \mid \langle p_1, p_2 \rangle \in \vec{v}\bigr] \\
        \texttt{EndSeqMap} &
            \state \tospec \mathbf{?} \sqsupseteq \bigl[p_2 \mid \langle p_1, p_2 \rangle \in \vec{v}\bigr] \\
        \bottomrule
    \end{tabu}
\end{example}

\paragraph{}
Most witness functions are domain-specific w.r.t.  the operator that they characterize.
However, once formulated in a module for a domain such as substring extraction, they can be reused by any DSL.
In our example, witness functions for most operators in \Cref{tbl:wfs:prose} (namely, all except \texttt{Map}) do not
depend on the domain of their parameters, and are therefore formulated generically, for any DSL.
Witness functions in \Cref{tbl:wfs:flashfill} hold only for their respective operators, but they do
not depend on the rest of the DSL in which these operators are used, provided the operator semantics is conformant with
its (strongly-typed) signature.
This property allows us to define witness functions as generally as possible in order to reuse the corresponding
operators in any conformant DSL.

\section{Deductive Search}
\label{sec:prose:algorithm}

A set of witness functions for all the parameters of an operator allows us to reduce the inductive synthesis problem
$\langle N, \spec\rangle$ to the synthesis subproblems for its parameters.
We introduce a simple non-conditional case first, and then proceed to complete presentation of the entire algorithm.

\begin{theorem}
    Let $N := F(N_1, \dots, N_k)$ be a rule in a DSL $\dsl$, and $\spec$ be a spec on $N$.
    Assume that $F$ has $k$ non-conditional witness functions $\omega_j(\spec) = \spec_j$,
    and $\vsa_j \models \spec_j$ for all $j = 1..k$ respectively.
    \begin{enumerate}[nosep]
        \item $\mathsf{Filter}(\joinCons(\vsa_1, \dots, \vsa_k), \spec) \models \spec$.
        \item If all $\omega_j$ are precise, then $\joinCons(\vsa_1, \dots, \vsa_k) \models \spec$.
    \end{enumerate}
    \label{thm:wf:noncond}
\end{theorem}
\begin{proof} \leavevmode
    \begin{enumerate}[nosep]
        \item By definition of $\mathsf{Filter}(\vsa, \spec)$.
        \item All $\omega_j$ deduce specs for $N_j$ given only the outer spec $\spec$, therefore they
            are independent from each other.
            Also, all $\omega_j$ are precise, therefore each $\vsa_j$ individually is necessary and sufficient to
            satisfy $\spec$. \qedhere
    \end{enumerate}
\end{proof}

\Cref{thm:wf:noncond} gives a straightforward recipe for synthesis of operators with independent parameters, such as
\dslinline|Pair($p_1$, $p_2$)|.
However, in most real-life cases operator parameters are dependent on each other.
Consider an operator \dslinline|Concat($atom$, $transform$)| from FlashFill, and a spec $\state \tospec s$.
It is possible to design individual witness functions $\omega_a$ and $\omega_t$ that return a disjunction $\spec_a$ of
prefixes of $s$ and a disjunction $\spec_t$ of suffixes of $s$, respectively.
Both of these witness functions individually are precise (i.e. sound and complete); however, there is no straightforward
way to combine recursive synthesis results $\vsa_a \models \spec_a$ and $\vsa_t \models \spec_t$ into a valid program
set for $\spec$.

In order to enable deductive search for dependent operator parameters, we apply \emph{skolemization}~\cite{modeltheory}.
Instead of deducing specs $\spec_a$ and $\spec_t$ that independently entail $\spec$, we deduce only one
independent spec (say, $\spec_a$), and then \emph{fix the value of ``$atom$''}.
For each fixed value of $atom$ a \emph{conditional witness function} $\omega_t(\spec \assuming atom = v)$ deduces a
spec $\spec_{t,v}$ that is a necessary and sufficient characterization for $\spec$.
Namely, $\spec_{t,v}$ in our example is $\state \tospec s[|v|..]$ (i.e. the remaining suffix) if $v$ is a prefix of~$s$,
or $\bot$ otherwise.

Skolemization splits the deduction into multiple independent branches, one per each value of $atom$.
These values are determined by VSA clustering: \mbox{$\clustering[\vsa_a] = \{v_1 \mapsto \vsa_a^1,
\dots, v_k \mapsto \vsa_a^k\}$}.
Within each branch, the program sets $\vsa_a^j$ and the corresponding $\vsa_t^j \models \spec_{t,v_j}$ are
independent, hence $\joinCons[\mathsf{Concat}](\vsa_a^j, \vsa_t^j) \models \spec$ by \Cref{thm:wf:noncond}.
The union of $k$ branch results constitutes a comprehensive set of all $\mathsf{Concat}$ programs that satisfy $\spec$.

\begin{defn}
    Let $N := F(N_1, \dots, N_k)$ be a rule in a DSL $\dsl$ with $k$ associated (possibly conditional) witness functions
    $\omega_1, \dots, \omega_k$.
    A \emph{dependency graph} of witness functions of $F$ is a directed graph $G(F) = \langle V, E\rangle$ where $V=
    \left\{N_1, \dots, N_k\right\}$, and $\langle N_i, N_j\rangle \in E$ iff $N_i$ is a prerequisite for $N_j$.
\end{defn}

A dependency graph can be thought of as a union of all possible Bayesian networks over parameters of $F$.
It is not a single Bayesian network because $G(F)$ may contain cycles: it is often possible to independently express
$N_i$ in terms
of $N_j$ as a witness function $w_i(\spec \assuming N_j = v)$ and $N_j$ in terms of $N_i$ as a different witness
function $w_j(\spec \assuming N_i = v)$.
One simple example of such phenomenon is \dslinline|Concat($atom$, $transform$)|: we showed above how to decompose its
inverse semantics into a witness function for prefixes and a conditional witness function for the suffix, but a
symmetrical decomposition into a witness function for suffixes and a conditional witness function for prefixes is also
possible.

\begin{figure}[p!]
    \begin{fullpage}
        \small
        \uwsinglespace
        \begin{algorithmic}[1]
            \Given{G(F)}{dependency graph of witness functions for the rule $F$}
            \Given{\spec}{specification for the rule $F$}
            \Functionx{LearnRule}{$G(F), \spec$}
            \State Permutation $\constraint \gets \mathsf{TopologicalSort}(G(F))$
            \State $\vsa \gets \bigvsaunion \bigl\{ \vsa' \bigmid \vsa' \in \Call{LearnPaths}{G(F), \spec, \constraint, 1, \varnothing} \bigr\}$
            \If{all witness functions in $G(F)$ are precise}
            \State \Return $\vsa$
            \Else
            \State \Return $\mathsf{Filter}(\vsa, \spec)$
            \EndIf
            \EndFunction
            \Statex
            \Given{\constraint}{permutation of the parameters of $F$}
            \Given{i}{index of a current deduced parameter in $\constraint$}
            \Given{Q}{a mapping of prerequisite values $\values_{k}$ and corresponding learnt program sets $\vsa_{k}$ on the current path}
            \Functionx{LearnPaths}{$G(F), \spec, \constraint, i, Q$}
            \If{$i > k$}
            \State Let $\vsa_1, \dots, \vsa_k$ be learnt program sets for $N_1, \dots, N_k$ in $Q$
            \State \Return $\left\{ \joinCons(\vsa_1, \dots, \vsa_k) \right\}$
            \EndIf
            \State $p \gets \constraint_i$ \Comment{Current iteration deduces the rule parameter $N_p$}
            \State Let $\omega_{p}(\spec \assuming N_{k_1} = \values_1, \dots, N_{k_m} = \values_m)$ be the witness
            function for $N_{p}$
            \Statex \Comment{Extract the prerequisite values for $N_{p}$ from the mapping $Q$}
            \State $\{\values_{k_1} \mapsto \vsa_{k_1}, \dots, \values_{k_m} \mapsto \vsa_{k_m}\} \gets Q[k_1, \dots, k_m]$
            \Statex \Comment{Deduce the spec for $N_{p}$ given $\spec$ and the prerequisites}
            \State $\spec_{p} \gets \omega_{p}(\spec \assuming N_{k_1} = \values_{k_1}, \dots, N_{k_m} = \values_{k_m})$ \label{alg:line:wf}
            \If{$\omega_p = \bot$}
            \Return $\emptyset$
            \EndIf
            \Statex \Comment{Recursively learn a valid program set $\vsa_{p} \models \spec_{p}$}
            \State $\vsa_{p} \gets \mathsf{Learn}(N_{p}, \spec_{p})$ \label{alg:line:sublearn}
            \Statex \Comment{If no other parameters depend on $N_{p}$, proceed without clustering}
            \If{$N_{p}$ is a leaf in $G(F)$}
            \State $Q' \gets Q[p := \top \mapsto \vsa_{p}]$
            \State \Return \Call{LearnPaths}{$G(F), \spec, \constraint, i+1, Q'$}
            \Statex \Comment{Otherwise cluster $\vsa_{p}$ on $\states$ and unite the results across branches}
            \Else
            \State $\states \gets$ the input states associated with $\spec$
            \ForAll{$\bigl(\values'_j \mapsto \vsa'_{s,j}\bigr) \in \clustering[\vsa_p][\states]$} \label{alg:line:clusters}
            \State $Q' \gets Q\bigl[s := \values'_j \mapsto \vsa'_{s,j}\bigr]$
            \State \Yield \textbf{all} \Call{LearnPaths}{$G(F), \spec, \constraint, i+1, Q'$}
            \EndFor
            \EndIf
            \EndFunction
        \end{algorithmic}
        % \end{mdframed}
        \caption{A learning procedure for the DSL rule $N := F(N_1, \dots, N_k)$ that uses $k$ conditional witness functions
        for $N_1, \dots, N_k$, expressed as a dependency graph $G(F)$.}
        \label{fig:prose:algorithm}
    \end{fullpage}
\end{figure}

\begin{theorem}
    Let $N := F(N_1, \dots, N_k)$ be a rule in a DSL $\dsl$, and $\spec$ be a spec on $N$.
    If there exists an acyclic spanning subgraph of $G(F)$ that includes each node with all its prerequisite edges, then
    there exists a polynomial procedure that constructs a valid program set $\vsa \models \spec$ from the valid
    parameter program sets $\vsa_j \models \spec_j$ for some choice of parameter specifications $\spec_j$.
    \label{thm:wf:cond}
\end{theorem}
\begin{proof}
    We define the learning procedure for $F$ in \Cref{fig:prose:algorithm} algorithmically.
    It recursively explores the dependency graph $G(F)$ in a topological order, maintaining a \emph{prerequisite path}
    $Q$ -- a set of
    parameters $N_j$ that have already been skolemized, together with their fixed bindings~$\values_j$ and valid program
    sets $\vsa_j$.
    In the prerequisite path, we maintain the invariant: \emph{for each program set $\vsa_j$ in the path, all programs
        in it produce the same values $\values_j$ on the provided input states~$\states$}.
    This allows each conditional witness function $\omega_{i}$ to deduce a spec $\spec_i$ for the current
    parameter~$N_i$ assuming the bound values $\values_{k_1}, \dots, \values_{k_s}$ for the prerequisites
    $N_{k_1}, \dots, N_{k_s}$ of $N_i$.

    The program sets in each path are valid for the subproblems deduced by applying witness functions.
    If all the witness functions in $G(F)$ are precise, then any combination of programs $P_1, \dots, P_k$ from these
    program sets yields a valid program $F(P_1, \dots, P_k)$ for $\spec$.
    If some witness functions are imprecise, then a filtered join of parameter program sets for each path is valid
    for $N$.
    Thus, the procedure in \Cref{fig:prose:algorithm} computes a valid program set $\vsa \models \spec$.
\end{proof}

\Cref{thm:wf:noncond,thm:wf:cond} give a \emph{constructive} definition of the refinement procedure that splits the
search space for $N$ into smaller parameter search spaces for $N_1,\dots,N_k$.
If the corresponding witness functions are precise, then \emph{every} combination of valid parameter programs from these
subspaces yields a valid program for the original synthesis problem.
Alternatively, if some of the accessible witness functions are imprecise, we use them to \emph{narrow down} the
parameter search space, and filter the constructed program set for validity.
The $\mathsf{Filter}$ operation (defined in \Cref{ch:vsa}) filters out inconsistent programs from $\vsa$ in time
proportional to $\clustering$.

\begin{sidewaysfigure}[p!]
    \begin{fullpage}
        \centering
        \def\sAtom{\text{\ttfamily atom}\,}
\def\sTransform{\text{\ttfamily transform}\,}
\hspace*{-6pt}
\begin{tikzpicture}[transform shape, remember picture,
                    deductive/.style={draw=SandyBrown, fill=SandyBrown!65, rounded corners=5pt,
                        align=left, inner sep=10pt},
                    entry/.style={draw=none, align=left},
                    fatarrow/.style={-{Triangle[scale=0.5]}, line width=8pt, draw=LightSkyBlue}]
    \large
    \uwsinglespace
    \tcbset{tile, size=title, on line, boxrule=0pt, colback=LightSkyBlue}
    \node[entry] (n-root) {$\langle \sTransform,\ \spec\rangle$};

    \node[entry, below right=2cm and -2.5cm of n-root] (n-concat)
        {$\langle\mathsf{Concat}\left(\sAtom,\, \sTransform\right),\ \spec\rangle$};
    \node[deductive, below=5pt of n-concat] (ded-concat) {%
        $\begin{aligned}
            &\spec_1 \coloneqq \omega_a\left(\spec\right) \\
            &\vsa_1 \coloneqq \subnode{learn-atom}{\tcbox{$\mathsf{Learn}\left(\sAtom, \spec_1\right)$}} \\
            &\{ o_1 \mapsto \vsa'_1,\, o_2 \mapsto \vsa'_2 \} \coloneqq \clustering[\vsa_1]
        \end{aligned}$};
    \draw[fatarrow] ($(n-root.south west) + (0.5cm,0)$) |- (n-concat.west);

    \node[entry, anchor=north west] at ($(ded-concat.south west) - (0,1.5cm)$) (n-atom)
        {$\langle \sAtom,\, \spec\rangle$};
    \draw[fatarrow] ($(n-root.south west) + (0.5cm,0)$) |- (n-atom.west);
    \node[entry, right=1cm of n-atom] (dots-atom) {$\dots$};
    \draw[fatarrow] (n-atom) -- (dots-atom);

    \node[entry, anchor=south] at ($(n-concat.north east) + (-0.7cm,1cm)$) (n-atom1)
        {$\langle \sAtom,\, \spec_1\rangle$};
    \draw[fatarrow, overlay] ($(learn-atom.east) - (11pt,0)$) -| ($(n-atom1.south east) - (15pt,0)$);

    \node[entry, left=1cm of n-atom1] (dots-atom1) {$\dots$};
    \draw[fatarrow] (n-atom1) -- (dots-atom1);

    \coordinate (c-right) at ($(n-atom1) + (4.5cm,0)$);
    \node[deductive, anchor=south west] at (c-right|-n-atom.south) (ded-cluster2) {%
        $\begin{aligned}
            &\spec_{22} \coloneqq \omega_t\left(\spec \assuming \sAtom = \stringliteral{20}\right) \\
            &\subnode{vsa-learn2}{$\color{Green}\vsa_{22}$} \coloneqq \subnode{learn-transform2}{\tcbox{%
                $\mathsf{Learn}\left(\sTransform,\, \spec_{22}\right)$}}
        \end{aligned}$};
    \draw[fatarrow, rounded corners=5pt, color=SandyBrown!65] ($(ded-concat.south east) - (0.5cm,0)$) |- (ded-cluster2);
    \node[entry, anchor=south east] at ($(ded-cluster2.north east) + (0, 0.6cm)$) (n-transform2)
        {$\langle \sTransform,\, \spec_{22} \rangle$};
    \draw[fatarrow] ($(learn-transform2.north east) - (14.2pt,11pt)$) -- ($(n-transform2.south east) - (13.9pt, -3pt)$);
    \node[entry, left=1cm of n-transform2] (dots-transform2) {$\dots$};
    \draw[fatarrow] (n-transform2) -- (dots-transform2);

    \coordinate(c-branch-x) at ($(n-atom1.south east) + (0.7cm,0)$);
    \coordinate (c-branch) at (c-branch-x|-ded-cluster2);
    \node[deductive, anchor=west] at (c-right|-n-concat.east) (ded-cluster1) {%
        $\begin{aligned}
            &\spec_{21} \coloneqq \omega_t\left(\spec \assuming \sAtom = \stringliteral{2}\right) \\
            &\subnode{vsa-learn1}{$\color{Green}\vsa_{21}$} \coloneqq \subnode{learn-transform1}{\tcbox{%
                $\mathsf{Learn}\left(\sTransform,\, \spec_{21}\right)$}}
        \end{aligned}$};
    \draw[fatarrow, rounded corners=5pt, color=SandyBrown!65] (c-branch) |- (ded-cluster1);
    \node[entry, anchor=south east] at ($(ded-cluster1.north east) + (0, 0.6cm)$) (n-transform1)
        {$\langle \sTransform,\, \spec_{21} \rangle$};
    \draw[fatarrow] ($(learn-transform1.north east) - (14.2pt,11pt)$) -- ($(n-transform1.south east) - (13.9pt, -3pt)$);
    \node[entry, left=1cm of n-transform1] (dots-transform1) {$\dots$};
    \draw[fatarrow] (n-transform1) -- (dots-transform1);

    \coordinate (c-branch-upper) at (c-branch|-ded-cluster1);
    \node[entry, color=Green, above right=0.3cm of c-branch] (n-vsa2) {$\vsa'_2$};
    \node[entry, below right=0.3cm of c-branch] (n-o2) {$o_2 = \stringliteral{20}$};
    \node[entry, color=Green, above right=0.3cm of c-branch-upper] (n-vsa1) {$\vsa'_1$};
    \node[entry, below right=0.3cm of c-branch-upper] (n-o1) {$o_1 = \stringliteral{2}$};

    \begin{scope}[on background layer]
        \node[anchor=north east, outer sep=3pt] at (n-o2.south west) (join2-sw) {};
        \node[above right=0.1cm of n-transform2.north east] (join2-ne) {};
        % \filldraw[color=PaleGreen!40, rounded corners=10pt] (join2-sw) rectangle (join2-ne);
        \node[entry, color=Green, anchor=north west, inner sep=6pt] at (join2-ne-|join2-sw) (join2-concat)
            {$\joinCons[\mathsf{Concat}]$};

        \node[anchor=north east, outer sep=3pt] at (n-o1.south west) (join1-sw) {};
        \node[above right=0.1cm of n-transform1.north east] (join1-ne) {};
        % \filldraw[color=PaleGreen!40, rounded corners=10pt] (join1-sw) rectangle (join1-ne);
        \node[entry, color=Green, anchor=north west, inner sep=6pt] at (join1-ne-|join1-sw) (join1-concat)
            {$\joinCons[\mathsf{Concat}]$};

        \node[entry, node font=\color{Green}, above left=0.2cm of c-branch] (union-inner) {$\bigvsaunion$};
        \node[entry, node font=\color{Green}, below right=0cm and .6cm of n-root.south west]
            (union-outer) {$\bigvsaunion$};
    \end{scope}
    \begin{scope}[every path/.style={-Stealth, dotted, thick, color=Green, rounded corners=10pt}]
        \draw (union-inner.north) |- (join2-concat.west);
        \draw (union-inner.north) |- (join1-concat.west);
        \draw (union-outer.south) |- (union-inner.west);
        \draw ($(join1-concat.south)!0.4!(join1-concat.south west)$) -- (n-vsa1);
        \draw ($(join1-concat.south)!0.5!(join1-concat.south east)$) to[bend left]
            ($(vsa-learn1.center)!0.5!(vsa-learn1.north)$);
        \draw ($(join2-concat.south)!0.4!(join2-concat.south west)$) -- (n-vsa2);
        \draw ($(join2-concat.south)!0.5!(join2-concat.south east)$) to[bend left]
            ($(vsa-learn2.center)!0.5!(vsa-learn2.north)$);
        \draw[sharp corners] (union-inner.west) -| (dots-atom.north);
    \end{scope}
\end{tikzpicture}

        \large
        \vspace*{-\baselineskip}
        \begin{equation*}
            \spec\colon\ \stringliteral{(202) 555-0126} \tospec \stringliteral{202}
            \quad
            \spec_1\colon\ \stringliteral{(202) 555-0126} \tospec \stringliteral{2} \,\vee\, \stringliteral{20}
            \quad
            \begin{aligned}[t]
                &\spec_{21}\colon\ \stringliteral{(202) 555-0126} \tospec \stringliteral{02} \\[-0.3\baselineskip]
                &\spec_{22}\colon\ \stringliteral{(202) 555-0126} \tospec \stringliteral{2}
            \end{aligned}
        \end{equation*}
        \caption{An illustrative diagram showing the outermost layer of deductive search for a given problem
        $\mathsf{Learn}\left(\mathtt{transform}\,,\, \stringliteral{(202) 555-0126} \tospec \stringliteral{202}\right)$.
        Solid blue arrows show the recursive calls of the search process (the search subtrees below the outermost layers
        are not expanded and shown as ``\dots'').
        Rounded orange blocks and arrows shows the nested iterations of the \textsc{LearnPaths} procedure from
        \Cref{fig:prose:algorithm}.
        Dotted green arrows show the VSA structure that is returned to the caller.}
        \label{fig:prose:algorithm:example}
    \end{fullpage}
\end{sidewaysfigure}

\begin{example}
    \Cref{fig:prose:algorithm:example} shows an illustrative diagram for the outermost layer of the learning process for
    the \dslinline|$transform$| symbol from $\ffdsl$ and a given spec $\spec = \stringliteral{(202) 555-0126} \tospec
    \stringliteral{202}$.
    First, the framework splits the search process into two branches: one for \dslinline|Concat($atom$, $transform$)|
    and one for \dslinline|$atom$|, according to the definition of \dslinline|$transform$| in $\ffdsl$.
    The figure shows the outermost layer of the first branch.

    The \textsc{LearnPaths} procedure first invokes the witness function $\omega_a$ for the first
    parameter $atom$ of the \dslinline|Concat| rule.
    It returns a necessary spec $\spec_1 = \state \tospec \stringliteral{2} \,\vee\,
    \stringliteral{20}$ for the $atom$ symbol (where $\state$ is the shared input state from $\spec$).
    The PROSE framework then recursively resolves this spec into a VSA $\vsa_1$ of all $atom$ programs that satisfy
    $\spec_1$.

    Next, the \textsc{LearnPaths} procedure clusters $\vsa_1$ and splits its further execution into two branches, one
    per each cluster of programs in $\vsa_1$ that give the same output on the given input state $\state$.
    There are two clusters: programs $\vsa'_1$ that return $o_1 = \stringliteral{2}$ and programs $\vsa'_2$ that return
    $o_2 = \stringliteral{20}$.
    For each of them the PROSE framework independently invokes a nested call to \textsc{LearnPaths} with the
    corresponding output binding for $atom$ (i.e. $o_1$ or $o_2$ respectively) recorded in the prerequisite path $Q$.
    Within each nested invocation, \textsc{LearnPaths} constructs a necessary and sufficient spec on the second
    parameter $transform$ of \dslinline|Concat| by invoking its conditional witness function $\omega_t$.
    It returns a spec $\spec_{21}\colon\ \state \tospec \stringliteral{02}$ for the branch with $o_1 =
    \stringliteral{2}$ and a spec $\spec_{22}\colon\ \state \tospec \stringliteral{2}$ for the branch with $o_2 =
    \stringliteral{20}$, respectively.
    The framework then recursively resolves them into the corresponding program sets $\vsa_{21}$ and $\vsa_{22}$.

    Since both witness functions $\omega_a$ and $\omega_t$ are precise, the final result returned from
    \textsc{LearnPaths} is the VSA $\bigvsaunion\left\{ \joinCons[\mathsf{Concat}]\left(\vsa'_1,\, \vsa_{21}\right),\,
    \joinCons[\mathsf{Concat}]\left(\vsa'_2,\, \vsa_{22}\right) \right\}$.
\end{example}

\begin{figure}[t]
    \centering
    \small
    \uwsinglespace
    \begin{subfigure}[t]{\textwidth}
        \subcaptionbox{\label{fig:topdown:spec}}{\hphantom{(a)}}\vspace{-1.5\baselineskip}
        \begin{mathpar}
            \infer{ N := F_1(N_1, \dots, N_k) \;|\; F_2(M_1, \dots, M_n) \\\\
            \text{\textsc{LearnRule}}(G(F_1), \spec) = \vsa_1\ \quad\ \text{\textsc{LearnRule}}(G(F_2), \spec) = \vsa_2}
            {\mathsf{Learn}(N, \spec) = \vsa_1 \vsaunion \vsa_2}
            \quad\infer{  }{\mathsf{Learn}(N, \top) = \dsl|_N}
            \\ \infer
            {\forall j = 1..2\colon \mathsf{Learn}(N, \spec_j) = \vsa_j}
            {\mathsf{Learn}(N, \spec_1 \wedge \spec_2) = \vsa_1 \cap \vsa_2}
            \quad \infer
            {\forall j = 1..2\colon \mathsf{Learn}(N, \spec_j) = \vsa_j}
            {\mathsf{Learn}(N, \spec_1 \vee \spec_2) = \vsa_1 \vsaunion \vsa_2}
            \quad\infer{  }{\mathsf{Learn}(N, \spec) = \mathsf{Filter}(\dsl|_N, \spec)}
            \\ \infer
            {\mathsf{Learn}(N, \spec_1) = \vsa \quad \spec_2 = \neg \langle \state, \constraint\rangle}
            {\mathsf{Learn}(N, \spec_1 \wedge \spec_2) = \mathsf{Filter}(\vsa, \spec_2)}
            \quad \infer
            {N := F(N_1, \dots, N_k) \\\\ \text{All witness functions in $G(F)$ accept $\spec$}}
            {\mathsf{Learn}(N, \spec) = \text{\textsc{LearnRule}}(G(F), \spec)} \\
        \end{mathpar}
    \end{subfigure}
    \begin{subfigure}[t]{\textwidth}
        \subcaptionbox{\label{fig:wf:syntax}}{\hphantom{(b)}}\vspace{-\baselineskip}
        \begin{mathpar}
            \quad\infer
            {N := \text{let } x = e_1 \text{ in } e_2}
            {\omega_{e_2}(\spec \assuming e_1 = v) = \state[x := v] \rightsquigarrow \spec}
            \quad\infer
            {N \text{ is a literal }}
            {\mathsf{Learn}(N, \state \rightsquigarrow v) = \left\{v\right\}}
            \\ \infer
            {N \text{ is a variable}}
            {\mathsf{Learn}(N, \state \rightsquigarrow \constraint) = \left\{N\right\} \text{if } \constraint(\state[N]) \text{ else } \emptyset}
        \end{mathpar}
    \end{subfigure}
    \caption{\textbf{(a)} Constructive inference rules for processing of boolean connectives in the inductive specifications $\spec$;
    \textbf{(b)} Witness functions and inference rules for common syntactic features of PROSE DSLs: \texttt{let} definitions, variables, and literals.}
    \label{fig:topdown:common}
\end{figure}

\paragraph{Handling Boolean Connectives}
Witness functions for DSL operators (such as the ones in \Cref{tbl:wfs:flashfill}) are typically defined on the
\emph{atomic} constraints (such as equality or subsequence predicates).
To complete the definition of deductive search, \Cref{fig:topdown:spec} gives inference rules for handling
of boolean connectives in a spec $\spec$.
Since a spec is defined as a NNF, we give the rules for handling conjunctions and disjunctions of specs, and
positive/negative literals.
These rules directly map to corresponding VSA operations:

\begin{theorem} \leavevmode
    \begin{enumerate}[nosep, label=(\arabic*)]
        \item $\vsa_1 \models \spec_1$ and $\vsa_2 \models \spec_2$ $\ \Longleftrightarrow\ $ $\vsa_1 \vsaunion \vsa_2 \models \spec_1 \vee \spec_2$.
        \item $\vsa_1 \models \spec_1$ and $\vsa_2 \models \spec_2$ $\ \Longleftrightarrow\ $ $\vsa_1 \cap \vsa_2 \models \spec_1 \wedge \spec_2$.
        \item $\vsa \models \spec_1$ $\ \Longleftrightarrow\ $ $\mathsf{Filter}(\vsa, \spec_2) \models \spec_1 \wedge \spec_2$.
    \end{enumerate}
    \label{thm:connectives:vsa}
\end{theorem}

Handling negative literals is more difficult.
They can only be efficiently resolved in two cases:
\begin{enumerate*}[label=(\alph*)]
    \item if a witness function supports the negated spec directly, or
    \item if the negative literal occurs in a conjunction with a positive literal, in which case we use the latter to
        generate a base set of candidate programs, which is then filtered to also satisfy the former
\end{enumerate*}.
If neither (a) nor (b) holds, the set of programs satisfying a negative literal is bounded only by the DSL.

% Our pre-defined generic witness functions in \Cref{fig:wf:library}, together with witness functions for common syntactic
% features of PROSE DSLs (\texttt{let} rules, variables, and literals) constitute the PROSE standard library.

\paragraph{Search Tactics}
\Cref{thm:wf:cond,thm:connectives:vsa,fig:topdown:common} entail a non-deterministic choice among numerous possible ways
to explore the program space deductively.
For instance, one can have many different witness functions for the same operator $F$ in $G(F)$, and they may deduce subproblems of
different complexity.
A specific exploration choice in the program space constitutes a \emph{search tactic} over a DSL.
We have identified several effective generic search tactics, with different advantages and disadvantages;
however, a comprehensive study on their selection is left for future work.

Consider a conjunctive problem $\mathsf{Learn}(N, \spec_1 \wedge \spec_2)$.
One possible way to solve it is given by \Cref{thm:connectives:vsa}: handle two conjuncts independently, producing VSAs
$\vsa_1$ and $\vsa_2$, and intersect them.
This approach has a drawback: the complexity of VSA intersection is quadratic.
Even if $\spec_1$ and $\spec_2$ are inconsistent (i.e. $\vsa_1 \cap \vsa_2 = \emptyset$), each conjunct individually may
be satisfiable.
In this case the unsatisfiability of the original problem is determined only after $T(\mathsf{Learn}(N, \spec_1)) +
T(\mathsf{Learn}(N, \spec_2)) + \mathcal{O}(\volume{\vsa_1} \cdot \volume{\vsa_2})$ time.

An alternative search tactic for conjunctive specs arises when $\spec_1$ and $\spec_2$ constrain \emph{different} input
states $\state_1$ and $\state_2$, respectively.
In this case each conjunct represents an independent ``world'', and witness functions can deduce subproblems in each
``world'' independently and concurrently.
PROSE applies witness functions to each conjunct in the specification in parallel, conjuncts the resulting parameter
specs, and makes a single recursive learning call.
Such ``parallel processing'' of conjuncts in the spec continues up to the terminal level, where the deduced sets of
concrete values for each terminal are intersected across all input states.\footnote{
    The ``parallel'' approach can also be thought of as a deduction over a new isomorphic DSL,
    in which operators (and witness functions) are straightforwardly lifted to accept \emph{tuples} of values
    instead of single values.}

The main benefit of this approach is that unsatisfiable branches are eliminated much sooner.
For instance, if among $m$ I/O examples one example is inconsistent with the rest, a parallel approach
approach discovers it as soon as the relevant DSL level is reached, whereas an intersection-based approach has to first
construct $m$ VSAs (of which one is empty) and intersect them.
Its main disadvantage is that in presence of disjunction the number of branches grows exponentially in a number of input
states in the specification.

\paragraph{Optimizations}
PROSE performs many practical optimizations in the algorithm in \Cref{fig:prose:algorithm}.
We parallelize the loop in \Cref{alg:line:clusters}, since it explores non-intersecting portions of the program space.
For ranked synthesis, we only calculate top $k$ programs for leaf nodes of $G(F)$, provided the ranking
function is monotonic.
We also cache synthesis results for every distinct learning subproblem $\langle N, \spec\rangle$, which makes
deductive search an instance of \emph{dynamic programming}.
This optimization is crucial for efficient synthesis of many common DSL operators, as we explain in more details in
\Cref{sec:prose:evaluation:casestudies}.

For bottom portions of the DSL we switch to enumerative
search~\cite{transit:protocols}, which in such conditions is more efficient than deduction, provided no constants need
to be synthesized.
In principle, every subproblem $\mathsf{Learn}(N, \spec)$ in PROSE can be solved by any sound strategy, not
necessarily deduction or enumerative search.
Possible alternatives include constraint solving or stochastic techniques~\cite{sygus}.

\section{Evaluation}
\label{sec:prose:evaluation}
Our evaluation of PROSE aims to answer two classes of questions: its \emph{applicability} and its \emph{performance}.
Applicability questions concern (a) our generalization of prior work in PBE
in terms of inductive specifications and witness functions; (b) generality of our library of witness functions;
(c) engineering usability of PROSE.
Performance questions concern the running time of synthesizers generated by PROSE, and the comparison of
PROSE to general-purpose non-inductive synthesizers, such as SyGuS~\cite{sygus}.

\subsection{Case Studies}
\label{sec:prose:evaluation:casestudies}
\Cref{tbl:prose:casestudies} summarizes our case studies: the prior works in inductive synthesis over numerous different
applications that we studied for evaluation of PROSE.
Of the \ref*{case:total} inductive synthesis tools we studied, \ref*{case:totaldc} can be cast as a special case of the
deductive search algorithm
methodology, which we verified by manually formulating corresponding witness functions for their algorithms.
In the other \pgfmathparse{int(\getrefnumber{case:total}-\getrefnumber{case:totaldc})}\pgfmathresult~tools, the
application domain is inductive synthesis, and our problem definition covers their application, but the original
technique is not an instance of deductive search: namely, it is enumerative
search~\cite{transit:protocols,magichaskeller} or constraint solving~\cite{quicksilver}.

\begin{table}[p!]
    \centering
    \small
    \newcounter{casenum}
    \setcounter{casenum}{-1}
    \begin{tabular}{>{\refstepcounter{casenum}}llcccc}
        \toprule
        \textbf{Project} & \textbf{Domain} & \textbf{Ded.} & \textbf{Impl.} & $\bm{\constraint}$ & $\bm{\spec'}$ \\
        \midrule
        \citet{flashfill} & String transformation & \yesmark & \yesmark & = & = \\
        \citet{flashextract} & Text extraction &  &  & $\sqsupset$ & $\sqsupset$ \\
        \citet{flashnormalize} & Text normalization &  &  & = & soft \\
        \citet{flashrelate} & Table normalization &  &  & = & = \\
        \label{case:totalreimpl}\citet{singh2012synthesizing} & Number transformation &  &  & = & = \\
        \midrule
        \citet{vldb12:semantic} & Semantic text editing & \yesmark & \nomark & = & = \\
        \citet{harris2011spreadsheet} & Table transformation &  &  & = & = \\
        \citet{andersen:procedural} & Algebra education &  &  & trace & = \\
        \citet{lau:smartedit} & Editor scripting &  &  & trace & = \\
        \citet{pldi15:swarat} & ADT transformation &  &  & = & = \\
        \citet{pldi15:osera} & ADT transformation &  &  & = & = \\
        \label{case:totaldc}\citet{miller:colorful} & Editor scripting &  &  & = & = \\
        \midrule
        \citet{transit:protocols} & Concurrent protocols & \nomark & \nomark & trace & N/A \\
        \citet{magichaskeller} & Haskell programs &  &  & = & N/A \\
        \label{case:total}\citet{quicksilver} & Relational queries &  &  & = & N/A \\
        \midrule
        \citet{raza2017automated} & Splitting of text into columns & \yesmark & \yesmark & = & = \\
        \citet{refazer} & Software refactoring & & & = & = \\
        \citet{gorinova2016end} & Reshaping of healthcare data & & & = & $\approx$ \\
        \textsf{Transformation.JSON} & Transformation of JSON trees &  &  & = & $\sqsupset$ \\
        \textsf{Extraction.JSON} & Extraction of data from JSON files &  &  & $\sqsupset$ & $\sqsupset$ \\
        \textsf{Matching.Text} & Data profiling/clustering &  &  & --- & = \\
        \label{case:totalwithnew}\textsf{Extraction.Web} & Web data extraction &  &  & $\sqsupset$ & $\sqsupset$ \\
        \bottomrule
    \end{tabular}
    \uwsinglespace
    \caption{Case studies of PROSE: prior works in inductive program synthesis.
        ``Ded.'' means ``Is it an instance of the deductive methodology?'',
        ``Impl.'' means ``Have we (re-)implemented it on top of PROSE?'', $\constraint$ is a top-level constraint kind,
        $\spec'$ lists notable intermediate constraint kinds (for the deductive techniques only).
        The bottommost section shows the new projects implemented on top of PROSE since its creation.}
    \label{tbl:prose:casestudies}
\end{table}

\begin{table}[t]
    \centering
    \begin{tabular}{lllll}
        \toprule
        \multicolumn{1}{c}{\multirow{2}{*}{\textbf{Project}}} & \multicolumn{2}{c}{\textbf{LOC}} &
            \multicolumn{2}{c}{\textbf{Development time}} \\
        \cmidrule{2-3}  \cmidrule{4-5}
        & Original & PROSE & Original & PROSE \\
        \midrule
        \citet{flashfill} & 12K & 3K & 9 months & 1 month \\
        \citet{flashextract} & 7K & 4K & 8 months & 1 month \\
        \citet{flashnormalize} & 17K & 2K & 7 months & 2 months \\
        \citet{flashrelate} & 5K & 2K & 8 months & 1 month \\
        \citet{singh2012synthesizing} & --- & 1K & --- & 2 months \\
        \midrule
        \citet{raza2017automated} & --- & 10K & --- & 2 months \\
        \citet{refazer} & & 6K & & 3 months \\
        \textsf{Transformation.JSON} & & 2K & & 1 month \\
        \textsf{Extraction.JSON} & & 3K & & 1 month \\
        \textsf{Matching.Text} & & 2K & & 3 months \\
        \textsf{Extraction.Web} & & 2.5K & & 1.5 months \\
        \bottomrule
    \end{tabular}
    \caption{Development data on the (re-)implemented projects.
        The cells marked with ``---'' either do not have an original implementation or we could not obtain historical
    data on them.}
    \label{tbl:prose:reimplementation}
\end{table}

Our industrial collaborators reimplemented \ref*{case:totalreimpl} existing systems
and created \pgfmathparse{int(\getrefnumber{case:totalwithnew}-\getrefnumber{case:total})}\pgfmathresult~new ones since
PROSE was created in \citeyear{flashmeta}.
We present data on these development efforts in \Cref{tbl:prose:reimplementation}.

\paragraph{Q1: How motivated is our generalization of inductive specification?}
Input-output examples is the most popular specification kind, observed in 12/\ref*{case:total} projects.
However, 3 projects require \emph{program traces} as their top-level specification, and 2 projects (1 prior) require
\emph{subsequences of program output}.
Boolean connectives such as $\vee$ and $\neg$ are omnipresent in subproblems across all \ref*{case:totaldc} projects
implemented using deductive search.

\paragraph{Q2: How applicable is our generic operator library?}
Most common operators across our case studies are string processing functions, due to the most popular domain being data
manipulation (11/\ref*{case:totalwithnew} projects).
Almost all projects include some version of learning conditional operators (equivalent to that of FlashFill).
List processing operators (e.g. $\mathsf{Map}$, $\mathsf{Filter}$) appear in 9/\ref*{case:totalwithnew} projects, often
without explicit realization by the original authors (for example, the awkwardly defined \textsf{Loop} operator in
FlashFill is actually a combination of \textsf{Concatenate} and \textsf{Map}).
\citet{pldi15:swarat} define an extensive library of synthesis strategies for list-processing operators in the
$\lambda^2$ project.
These synthesis strategies are isomorphic to FlashExtract witness functions; both approaches can be cast as instances of
deductive search (see \Cref{ch:related} for detailed comparison).

\paragraph{Q3: How usable is PROSE?}
\Cref{tbl:prose:reimplementation} presents some development stats on the projects that were reimplemented.
In all cases, PROSE-based implementations were shorter, cleaner, more stable and extensible.
The reason is that with PROSE, our collaborators did not concern themselves with tricky details of synthesis
algorithms, since they were implemented once and for all, as in \Cref{sec:prose:algorithm}.
Instead, they focused only on domain-specific witness functions, for which design, implementation, and maintenance are
much easier.
Notably, in case of the FlashRelate~\cite{flashrelate} reimplementation and \textsf{Extraction.Web}, our collaborators
did not have any experience in program synthesis.

The development time in \Cref{tbl:prose:reimplementation} includes the time required for an implementation to mature
(i.e. cover the required use cases), which required multiple experiments with DSLs.
With PROSE, various improvements over DSLs were possible on a daily basis.
PROSE also allowed our collaborators to discover optimizations not present in the original implementations.
We share some anecdotes of PROSE simplifying synthesizer development below.

\begin{scenario}
    One of the main algorithmic insights of FlashFill is synthesis of $\mathsf{Concat}(e_1, \dots, e_k)$ expressions
    using \emph{DAG program sharing}.
    A DAG over the positions in the output string $s$ is maintained, each edge $s[i:j]$ annotated with a
    set of programs that output this substring on a given state $\state$.
    Most of the formalism in the paper and code in their implementation is spent on describing and performing operations
    on such a DAG.
    In PROSE, the same grammar symbol is instead defined through a recursive binary operator: $f \coloneq e \,|\,
    \mathsf{Concat}(e, f)$.
    The witness function for $e$ in \textsf{Concat} constructs $\spec'$ as a disjunction of all prefixes of
    the output string in $\spec$.
    The property for $f$ is conditional on $e$ and simply selects the suffix of the output string after the given prefix
    $\semantics{e}{\state}$.
    Since PROSE caches the results of learning calls $\langle f, \spec \rangle$ for same $\spec$s, the tree of
    recursive $\mathsf{Learn}(f, \spec)$ calls becomes a DAG, as shown in \Cref{fig:prose:evaluation:dag}.
    This is \emph{the same DAG} as in FlashFill -- but with PROSE, it arises implicitly and at no cost.
    Moreover, it becomes obvious now that DAG sharing happens for any foldable operator, e.g. \textsf{ITE}, $\wedge$,
    $\vee$, sequential statements.
    \label{sc:dag}
\end{scenario}

\begin{figure}
    \newcommand{\sproblem}[2]{\ensuremath{\langle #1,\, \stringliteral{#2}\rangle}}
    \newcommand{\sproblemDisj}[3]{\ensuremath{\langle #1,\, \stringliteral{#2} \vee \stringliteral{#3}\rangle}}
    \centering
    \begin{tikzpicture}[f/.style={draw, rounded corners, inner sep=7pt, fill=PowderBlue},
                        e/.style={draw, sharp corners, shape=rectangle, inner sep=7pt, fill=LightGoldenrodYellow,
                                     minimum width=2.2cm},
                        concat/.style={},
                        every path/.style={-Stealth, rounded corners=10pt}]
        \node[f] (t202) {\sproblem{f}{202}};
        \node[concat, anchor=west, below=0.5cm of t202.south east] (conc202) {\textsf{Concat}};
        \draw (t202.west) |- (conc202);
        \node[f, anchor=west] at ($(conc202.south) + (1cm, -2cm)$) (t02) {\sproblem{f}{02}};
        \draw (conc202.south) |- (t02);
        \node[concat, anchor=west, below=0.5cm of t02.south east] (conc02) {\textsf{Concat}};
        \draw (t02.west) |- (conc02);
        \node[f, below right=0.5cm of conc02] (t2) {\sproblem{f}{2}};
        \draw (conc02.south) |- ($(t2.north west)!0.33!(t2.south west)$);
        \draw ($(conc202.south west)!0.5!(conc202.south)$) |- ($(t2.south west)!0.33!(t2.north west)$);

        \node[e, right=11.5cm of t202] (a202) {\sproblem{e}{202}};
        \node[e, below=0.25cm of a202] (a20) {\sproblem{e}{20}};
        \node[e, below=0.25cm of a20] (a2) {\sproblem{e}{2}};
        \node[e, below=0.25cm of a2] (a02) {\sproblem{e}{02}};
        \node[e, below=0.25cm of a02] (a0) {\sproblem{e}{0}};

        \draw[dashed] (t202) -> (a202);
        \node[e, right=4.5cm of conc202] (adisj) {\sproblemDisj{e}{2}{20}};
        \draw[dashed] (conc202) -> (adisj);
        \draw[dashed] (adisj) -> (a20);
        \draw[dashed] (adisj) |- ($(a2.north west)!0.33!(a2.south west)$);
        \draw[dashed] (t02) -> (a02);
        \draw[dashed] (conc02) -> (a0);
        \coordinate (target) at ($(a2.south west)!0.33!(a2.north west)$);
        \coordinate (cut) at (adisj.east|-target);
        \draw[dashed] (t2.east) -| (cut) -> (target);
    \end{tikzpicture}
    \caption{A DAG of recursive calls that arises during the deductive search process for a recursive binary operator
        $f \coloneq e \;|\; \mathsf{Concat}(e,\, f)$.
        As described in \Cref{sc:dag}, it is isomorphic to an explicitly maintained DAG of substring programs in the
        original FlashFill implementation~\cite{flashfill}.}
    \label{fig:prose:evaluation:dag}
\end{figure}

\begin{scenario}
    During reimplementation of FlashFill, a new operator was added to its substring extraction logic: \emph{relative
    positioning},
    which defines the right boundary of a substring depending on the value of its left boundary.
    For example, it enables extracting substrings as in ``ten characters after the first digit''.
    This extension simply involved adding three \texttt{let} rules in the DSL, which (a) define the left boundary
    position using existing operators; (b) cut the suffix starting from that position; (c) define the right boundary in
    the suffix.
    While such an extension in the original FlashFill implementation would consume a couple of weeks, in PROSE it
    took only a few minutes.
\end{scenario}

\begin{scenario}
    A CSS selector is a function $\mathsf{Document} \to \mathsf{Set}\langle \mathsf{DOMNode}\rangle$.
    It is a path specification for a DOM node where each element in the path is a predicate on the corresponding
    ancestor (i.e.  the ancestor's tag or its class), and each edge in the path descends to all children of the
    preceding element that satisfy a certain property~\cite{css3selectors}.

    A spec synthesis of CSS selectors is a subset of selected DOM nodes.
    Using an enumerative search for this problem induces an exponential blowup: it starts with an input state (an HTML
    document) and iteratively constructs all possible CSS selectors.
    Since they may select arbitrary subsets of the DOM tree, the resulting search is infeasible.

    In contrast, a deductive approach starts with an \emph{output} (a set of nodes), and deduces examples for the
    intermediate subexpressions (prefixes of the desired CSS selector).
    This process follows the DOM tree \emph{upwards}, instead of \emph{downwards}, and therefore is by construction
    finite.
    Moreover, the number of deduction steps is bounded by the tree depth.
\end{scenario}

\begin{figure}[p!]
    \centering
    \begin{tikzpicture}
        \node[anchor=south west, inner sep=0] (img) at (0,0) {\includegraphics[width=\textwidth]{figures/perf}};
        \begin{scope}[x={(img.south east)}, y={(img.north west)}]
            \node[inner sep=10pt, rounded corners, fill=Tan] at (0.3, 1) {FlashFill};
            \node[inner sep=10pt, rounded corners, fill=Tan] at (0.78, 1) {FlashExtract};
            \node[fill=white] at (0.3, 0.02) { Number of examples };
            \node[fill=white] at (0.78, 0.02) { Number of examples };
            \node[fill=white, rotate=90] at (0.03, 0.52) { Average learning time (Log) };
        \end{scope}
    \end{tikzpicture}
    \caption{Performance evaluation of the reimplementations of FlashFill~\cite{flashfill} and
    FlashExtract~\cite{flashextract} on top of the PROSE framework.
    Each dot shows average learning time per iteration on a given scenario.
    The scenarios are clustered by the number of examples (iterations) required to complete the task.
    \emph{Left:} 531 FlashFill scenarios.
    \emph{Right:} 6464 FlashExtract scenarios.}
    \label{fig:prose:evaluation:perf}
\end{figure}

\subsection{Experiments}
\label{sec:prose:evaluation:experiments}

\paragraph{Performance \& Number of examples}
\Cref{fig:prose:evaluation:perf} shows performance and the number of examples of our FlashFill and FlashExtract
reimplementations on top of the PROSE framework.
The overall performance is comparable to that of the original system, even though the implementations differ
drastically.
For example, the runtime of the original implementation of FlashExtract varies from 0.1 to 4 sec, with a median of 0.3
sec~\cite{flashextract}.
The new implementation (despite being more expressive and built on a general framework) has a runtime of
$0.5-3$x the original implementation, with a median of 0.6 sec.
This performance is sufficient for the PROSE-based implementation to be successfully used in industry instead of the
original one.

\paragraph{VSA Volume}
There is no good theoretical bound on the time of VSA clustering (the most time-consuming operation in the deductive
search).
However, it is evident that the output VSA volume is proportional to the clustering time.
Thus, to evaluate it, we measured the VSA volume on our real-life benchmark suite.
As \Cref{fig:prose:evaluation:volume} shows, even for large inputs it never exceeds $8000$ nodes (thus explaining
efficient runtime), whereas VSA size (i.e. number of learned programs) may approach $10^{13}$.

\begin{figure}[t]
    \centering
    \includegraphics[width=1.0\linewidth]{figures/vsa-volume}
    \uwsinglespace
    \caption{The relationship between VSA volume and VSA size (i.e. number of programs) for the complete VSAs learned to
    solve 6464 real-life FlashExtract scenarios.}
    \label{fig:prose:evaluation:volume}
\end{figure}

\section{Strengths and Limitations}
\label{sec:prose:discussion}
The methodology of deductive search that lies at the core of PROSE works best under the following conditions:

\paragraph{Decidability}
A majority of the DSL should be characterized by witness functions, capturing a subset of inverse semantics of the DSL
operators.

An example of an operator that cannot be characterized by any witness function is an integral multivariate polynomial
$\mathsf{Poly}(a_0, \dots, a_k, X_1, \dots, X_n)$.
Here $a_0, \dots, a_k$ are integer polynomial coefficients, which are input variables in the DSL,
and $X_1, \dots, X_n$ are integer nonterminals in the DSL.
Given a specification $\spec = (a_0, \dots, a_k) \tospec y$ stating that a specific $\mathsf{Poly}$ executed with
coefficients $a_0, \dots, a_k$ evaluated to $y$ on \emph{some} $X_1, \dots, X_n$, a witness function~$\omega_j$ has to
find a set of possible values for $X_j$.
This requires finding roots of a multivariate integral polynomial, which is undecidable.

\paragraph{Deduction}
Witness functions should not introduce many disjunctions.
Each disjunct (assuming it can be materialized by at least one program) starts a new deduction branch.
In certain domains this problem can only be efficiently solved with a corresponding SMT solver.

Consider the bitwise operator $\mathsf{BitOr}\colon (\mathsf{Bit32}, \mathsf{Bit32}) \to \mathsf{Bit32}$.
Given a specification $\state \tospec b$ where $b\colon \mathsf{Bit32}$, witness functions for $\mathsf{BitOr}$
have to construct each possible pair of bitvectors $\langle b_1, b_2\rangle$ such that $\mathsf{BitOr}(b_1, b_2) = b$.
If $b = 2^{32} - 1$, there exist $3^{32}$ such pairs.
A deduction over $3^{32}$ branches is infeasible.

\paragraph{Performance}
Witness functions should be efficient, preferably polynomial in low degrees over the specification size.

Consider the multiplication operator $\mathsf{Mul}\colon (\mathsf{Int}, \mathsf{Int}) \to \mathsf{Int}$.
Given a specification $\state \tospec n$ with a multiplication result, a witness function for $\mathsf{Mul}$
has to factor $n$.
This problem is decidable, and the number of possible results is at most $\mathcal{O}(\log n)$, but the factoring itself
is infeasible for large $n$.

\paragraph{}
\indent All counterexamples above feature real-life operators, which commonly arise in embedded systems, control theory,
and other domains.
The best known synthesis strategies for them are based on specialized SMT solvers~\cite{sygus}.
On the other hand, to our knowledge PROSE is the \emph{only} synthesis strategy when the following (also real-life)
conditions hold:
\begin{itemize}[nosep]
    \item The programs may contain domain-specific constants.
    \item The DSL contains arbitrary executable operators that manipulate domain-specific objects with rich semantics.
    \item The specifications are inherently ambiguous, and resolving user's intent requires learning a set of valid
        programs to enable ranking or additional user interaction.
    \item The engineering and maintenance cost of a PBE\hyp{}based tool is limited by industrial budget and available
        developers.
\end{itemize}


