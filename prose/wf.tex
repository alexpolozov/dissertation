\section{Witness Functions}
\label{sec:prose:wf}
As described above, a \emph{witness function} is a generalization of inverse semantics for an operator.
In other words, it is a \emph{problem reduction logic}, which deduces a necessary (or sufficient) spec on the operator's
parameters given a desired spec on the operator's output.
In this section, I define witness functions formally and give various examples of their usages in the existing PBE
technologies.

\begin{defn}[Witness function]
    Let $F\left(N_1,\dots,N_k\right)$ be an operator in a DSL $\dsl$.
    A \emph{witness function} of $F$ for $N_j$ is a function $\omega_{j}\left(\spec\right)$ that deduces a
    \emph{necessary} spec $\spec_j$ on $N_j$ given a spec $\spec$ on $F\left(N_1, \dots, N_k\right)$.
    Formally, $\omega_j\left(\spec\right) = \spec_j$ iff the following implication holds:\footnote{All free variables
        are universally quantified unless otherwise specified.}
    \begin{equation*}
        F\left(N_1, \dots, N_k\right) \models \spec \qquad\Longrightarrow\qquad N_j \models \spec_j.
    \end{equation*}
\end{defn}

\begin{defn}[Precise witness function]
    A witness function $\omega_j$ of $F\left(N_1, \dots, N_k\right)$ for $N_j$ is \emph{precise} if its deduced spec is
    \emph{necessary and sufficient}.
    Formally, $\omega_j\left(\spec\right) = \spec_j$ is precise iff
    \begin{equation*}
        N_j \models \spec_j \qquad\Longleftrightarrow\qquad
        \exists\, N_1, \dots, N_{j-1}, N_{j+1}, \dots, N_k\colon\ F\left(N_1, \dots, N_k\right) \models \spec.
    \end{equation*}
\end{defn}

\begin{defn}[Conditional witness function]
    A \emph{(precise) conditional witness function} of $F\left(N_1, \dots, N_k\right)$ for $N_j$ is a function
    $\omega_j\left(\spec \assuming N_{k_1} = v_1, \dots, N_{k_s} = v_s\right)$ that deduces a necessary (and sufficient)
    spec $\spec_j$ on $N_j$ given a spec $\spec$ on $F\left(N_1, \dots, N_k\right)$ under the assumption that a subset
    of other parameters $N_{k_1}, \dots, N_{k_s}$ of $F$ (called \emph{prerequisites}) have fixed values $v_1, \dots,
    v_k$.
    Formally, $\omega_j\left(\spec \assuming N_t = v_t\right) = \spec_j$ iff the following implication holds:
    \begin{equation*}
        F\left(N_1, \dots, N_{t-1}, v_t, N_{t+1}, \dots, N_k\right) \models \spec
        \qquad\Longrightarrow\qquad
        N_j \models \spec_j.
    \end{equation*}
\end{defn}

\begin{sidewaystable}[p!]
    \begin{fullpage}
        \centering
        % \uwsinglespace
        \small
        \begin{tabu}{lX[-1$l]X[-1$l]X[-1$l]>{\hspace*{1.2em}}X[-3$l]}
            \toprule
            \textbf{Operator} & \textbf{Input spec} & \textbf{Parameter} & \textbf{Prerequisites} & \textbf{Output
            spec} \\
            \midrule
            \dslinline|Concat($atom$,\ $transform$)| & \state \tospec w & atom & &
                \state \tospec \bigvee\limits_{j=1}^{|w|-1} w[0..j] \\
                \cmidrule{3-5}
                & & transform & atom = v & \state \tospec w[|v|..] \\
            \midrule
            \dslinline|ConstStr($s$)| & \state \tospec w & s & & \state \tospec w \\
            \midrule
            \dslinline|let\ $x\ $ = $\ \tikzmark{mark:b:Bind}$std.Kth($inputs$,\ $k$)$\tikzmark{mark:e:Bind}$| &
              \state \tospec w & binding & &
              \state \tospec \bigvee\limits_{\mathclap{j\colon w \text{ occurs in } v_j}}\ v_j \\[3pt]
              \cmidrule{3-5}
          \qquad\dslinline|in Substring($x$,\ $pp$)| & & pp & x = v & \state \tospec
              \bigvee\limits_{\mathclap{\substack{w \text{ occurs in } v \\ \text{ at position } l\ \ }}}
              \ \langle l, l + |w|\rangle \\
            \midrule
            \dslinline|AbsPos($x$,\ $k$)| & \state \tospec c & k & x = v & \state \tospec c \vee c - |v| - 1 \\
            \midrule
            \dslinline|RegexPos($x$, $\ \tikzmark{mark:b:RR}$std.Pair($r$,\ $r$)$\tikzmark{mark:e:RR}$,\ $k$)| &
              \state \tospec c & rr & x = v &
              \state \tospec \bigvee\limits_{\mathclap{\substack{
                          \langle r_1, r_2 \rangle\colon r_1 \text{ matches left of } c \\
                          \hspace*{2.5em}\wedge\ r_2 \text{ matches right of } c}}}\ \langle r_1, r_2 \rangle \\
              \cmidrule{3-5}
              & & k & \begin{aligned} &x = v,\\[0pt] &rr = \langle r_1, r_2 \rangle \end{aligned} &
                  \begin{aligned}
                      &\state \tospec j \vee j - |\vec{c}| - 1 \quad\text{ where } \\
                      &\quad \vec{c} \text{ are the matches of } \langle r_1, r_2 \rangle \text{ in } v, \\
                      &\quad j \text{ is an index of } c \text{ in } \vec{c}
                  \end{aligned} \\
            \bottomrule
        \end{tabu}
        \begin{tikzpicture}[overlay, remember picture]
            \draw[decorate, decoration={brace, mirror}, transform canvas={yshift=-0.3em}]
                (pic cs:mark:b:Bind) -- (pic cs:mark:e:Bind) node[midway, below] {$binding$};
        \end{tikzpicture}
        \begin{tikzpicture}[overlay, remember picture]
            \draw[decorate, decoration={brace, mirror}, transform canvas={yshift=-0.3em}]
                (pic cs:mark:b:RR) -- (pic cs:mark:e:RR) node[midway, below] {$rr$};
        \end{tikzpicture}
        \caption{Witness functions for various FlashFill operators.}
        \label{tbl:wfs:flashfill}
    \end{fullpage}
\end{sidewaystable}

\begin{sidewaystable}[p!]
    \begin{fullpage}
        \centering
        % \uwsinglespace
        \small
        \begin{tabu}{lX[-1$l]X[-1$l]X[-1$l]X[-3$l]}
            \toprule
            \textbf{Operator} & \textbf{Input spec} & \textbf{Parameter} & \textbf{Prerequisites} & \textbf{Output
            spec} \\
            \midrule
            \dslinline|Kth($xs$,\ $k$)| & \state \tospec w & xs & & \state \tospec [\dots, w, \dots] \\
            \cmidrule{3-5}
            & & k & xs = \vec{v} & \state \tospec \bigvee\limits_{v_j = w} j \\
            \midrule
            \dslinline|Pair(|$p_1$\dslinline|,\ |$p_2$\dslinline|)| & \state \tospec \langle v_1, v_2 \rangle & p_j & &
                \state \tospec v_j \\
            \midrule
            \dslinline|Map($F$,\ $L$)| & \state \tospec \mathbf{?} \sqsupseteq \vec{\ell} \text{ as a prefix } & F &
                L = \vec{v} & \state \tospec f \text{ s.t. } \bigwedge\limits_{i=1}^{|\vec{\ell}|} f(v_i) = \ell_i \\
            \midrule
            $\fun{x}\ b$ & \state \tospec f \text{ s.t. } f(v) = y & b & & \state[x \coloneq v] \tospec y \\
            \midrule
            \dslinline|Filter($P$,\ $L$)| & \state \tospec \mathbf{?} \sqsupseteq \vec{\ell} & L & &
                \state \tospec \mathbf{?} \sqsupseteq \vec{\ell} \\
            \cmidrule{3-5}
            & & P & L = \vec{v} & \state \tospec f \text{ s.t. }
                \bigwedge\limits_{i=1}^{|\vec{v}|} f(\ell_i) = \bigl[v_i \in \vec{\ell}\bigr] \\
            \midrule
            \dslinline|FilterInt($i$,\ $k$,\ $L$)| & \state \tospec \mathbf{?} \sqsupseteq \vec{\ell} & L & &
                \state \tospec \mathbf{?} \sqsupseteq \vec{\ell} \\
            \cmidrule{3-5}
            & & i & L = \vec{v} & \state \tospec \text{ index of } \ell_1 \text{ in } \vec{v} \\
            \cmidrule{3-5}
            & & k & L = \vec{v} &
                \begin{aligned}
                    &\state \tospec \bigvee\limits_{d\;|\; g} d \quad\text{ where } \\
                    &\quad g = \mathsf{gcd}\bigl(\Delta_1, \dots, \Delta_{|\vec{\ell}|-1}\bigr), \\
                    &\quad \Delta_i = p_{i+1} - p_i, \\
                    &\quad p_j \text{ is an index of } \ell_j \text{ in } \vec{v}
                \end{aligned} \\
            \bottomrule
        \end{tabu}
        \caption{Witness functions for some operators from the standard library of PROSE.}
        \label{tbl:wfs:prose}
    \end{fullpage}
\end{sidewaystable}



\begin{example}
    \Cref{tbl:wfs:flashfill} shows witness functions for all FlashFill operators from \Cref{fig:dsl:flashfill}:
    \begin{itemize}[nosep]
        \item A \dslinline|Concat($atom$, $transform$)| expression returns $w$ iff $atom$ returns some prefix of $w$.
            In addition, assuming that $atom$ returns $v$, $transform$ mush return the remaining suffix of~$w$ after the
            end of $v$.
        \item A \dslinline|ConstStr($s$)| expression returns $w$ iff $s$ is equal to $w$.
        \item An expression ``\dslinline|let $x$ = std.Kth($inputs$, $k$) in $\dots$|'' returns $w$ iff $x$ is bound to
            an element of $inputs$ that has $w$ as a substring.
        \item A \dslinline|Substring($x$, $pp$)| expression returns $w$ (assuming that $x$ returns $v$) iff $pp$ returns
            a position span of any occurrence of $w$ in $v$ as a substring.
        \item An \dslinline|AbsPos($x$, $k$)| expression returns $c$ (assuming that $x$ returns $v$) iff $k$ is equal to
            either~$c$ or $c-|v|-1$ (since $k$ may represent a left or right offset depending on its sign).
        \item An expression \dslinline|RegexPos($x$, $rr$, $k$)| returns $c$ (assuming that $x$ returns $v$) iff $rr$ is
            equal to any pair of regular expressions that matches the boundaries of position $c$ in the string~$v$.
            In addition, assuming that $rr$ is equal to $\langle r_1, r_2\rangle$, $P$ returns $c$ iff $k$ is equal to
            a index of $c$ (from the left or right) among all matches of $\langle r_1, r_2\rangle$ in $v$.
    \end{itemize}
    \label{ex:wf:flashfill}
\end{example}

Most witness functions are domain-specific w.r.t.  the operator that they characterize.
However, once formulated in a module for a domain such as substring extraction, they can be reused by any DSL.
In our example, witness functions for $\mathsf{Pair}$ and $\mathsf{Kth}$ operators in \Cref{tbl:wfs:prose} do not
depend on the domain of their parameters, and are therefore formulated generically, for any DSL.
Witness functions in \Cref{tbl:wfs:flashfill} hold only for their respective operators, but they do
not depend on the rest of the DSL in which these operators are used, provided the operator semantics is conformant with
its (strongly-typed) signature.
This property allows us to define witness functions as generally as possible in order to reuse the corresponding
operators in any conformant DSL.
