\PassOptionsToPackage{numbers, sort}{natbib}
\documentclass[final, 12pt, proquest, natbib]{uwthesis}
\usepackage[T1]{fontenc}
\usepackage[utf8]{inputenc}
\usepackage{microtype}
\usepackage{textcomp}

% MATH
% To prevent the ``too many alphabets'' error
\newcommand{\hmmax}{0}
\newcommand{\bmmax}{0}
\usepackage{amsmath, amsthm, amsfonts, amssymb}
\usepackage{thmtools}
\usepackage{mathtools}
\usepackage{mathpartir}

%%% FONTS
\usepackage[scaled=.98, sups, osf]{XCharter}
\usepackage[scaled=1.04, varqu, varl]{inconsolata}
\usepackage[scaled]{berasans}
\usepackage[xcharter, cmbraces, varg, vvarbb, scaled=1.07,
            subscriptcorrection]{newtxmath}
\usepackage[cal=boondoxo]{mathalfa}
\usepackage{pifont}
\renewcommand*\copyright{{%
    \fontfamily{bch}\selectfont\textcopyright}}
\DisableLigatures{encoding=T1, family=tt*}
\let\emptyset\emptysetAlt

%%% GRAPHICS
\usepackage{graphicx}
\usepackage[usenames, svgnames]{xcolor}
\usepackage[hypcap=true]{subcaption}
\usepackage[dot, tikz, math, cache, autosize]{dot2texi}
\usepackage{tcolorbox}
\usepackage{adjustbox}
\usepackage{tikz}
\usepackage{pgfplots}
\usepackage{tikzscale}
\usetikzlibrary{shadings, shapes, arrows, tikzmark, decorations.pathreplacing,
                fit, positioning, arrows.spaced, arrows.meta, calc, backgrounds,
                quotes}
\tcbuselibrary{raster, skins}
\tcbset{sharp corners=all, boxrule=0.1mm, colback=black!3!white}
\tikzset{
    every picture/.append style={>=latex},
    join node/.style={rounded corners, fill=LightGray!70, draw=Gray},
    vsa node/.style={fill=blue!15, draw=RoyalBlue}
}

%%% TYPESETTING
\usepackage[inline]{enumitem}
\usepackage{booktabs}
\usepackage[super]{nth}
\usepackage{bm}
\usepackage[figuresright]{rotating}

%%% UTILITIES
\usepackage{refcount}
\usepackage{tabu, multirow}
\usepackage{import}
\usepackage{adjustbox}
\usepackage{relsize}
\usepackage{scalerel}
\usepackage{twoopt}
\usepackage{etoolbox}
\usepackage{xspace}
\usepackage{hyphenat}
\usepackage[xspace]{ellipsis}
\usepackage{syntax}
\usepackage{soulpos}
\usepackage[color=red!55, textsize=footnotesize, textwidth=1.3cm,
            shadow, prependcaption, obeyFinal, colorinlistoftodos]{todonotes}

%%% ALGORITHMS
\usepackage{algorithm}
\usepackage[noend]{algpseudocode}
\usepackage{listings}
\lstset{
    basicstyle = \small\ttfamily,
    stringstyle = \color{Brown},
    keywordstyle = \color{MediumBlue},
    commentstyle = \itshape\rmfamily\color{Green},
    showstringspaces = false,
    columns = flexible,
	breaklines = true,
	texcl = true,
	mathescape = true,
    tabsize = 4,
}
\lstalias[]{csharp}[Sharp]{C}
\lstdefinelanguage{dsl}{
    morekeywords = {language,feature,using,semantics,learners,int,string,@input,@start,values,let,in,@ref,@values,Tuple,@extern,@output,namespace,bool,std,@id,grammar},
    otherkeywords = {:=,=>,:,[]},
    sensitive = true,
    morecomment = [l]{//},
    morestring = [b]',
}
\definecolor{linenum-gray}{HTML}{888888}
\renewcommand{\algorithmiccomment}[1]{ {\itshape \color{Green} \textsc{//} #1}}
\algnewcommand\algorithmicyield{\textbf{yield}}
\algnewcommand\Yield{\algorithmicyield\ \algorithmicreturn\ }
\newcommand{\NoNumber}{\def\alglinenumber##1{}}
\newcommand{\WithNumber}{\def\alglinenumber##1{\sf\scriptsize\color{linenum-gray}##1:\hspace{-11pt}}}
\newcommand{\Functionx}[2]{\NoNumber \Function{#1}{#2} \addtocounter{ALG@line}{-1} \WithNumber}
\let\oldStatex\Statex
\renewcommand{\Statex}{\oldStatex \hspace{\algorithmicindent}}
\algnewcommand\algorithmicinput{\textbf{Input}}
\algnewcommand\Given[2]{\oldStatex \algorithmicinput\ \ensuremath{\bm{#1}}: \begin{varwidth}[t]{0.80\linewidth} #2 \end{varwidth}}
\newcommand*{\Let}[2]{\State #1 $\gets$ \parbox[t]{\linewidth-\algorithmicindent}{#2\strut}}

%%% OPTIONS
% Whether to color hyperlinks in the PDF
\newtoggle{colorlinks}
\toggletrue{colorlinks}

%%% HYPERLINKS
\usepackage{url}
\usepackage[breaklinks=true, colorlinks=true, pdfauthor={Oleksandr Polozov}]{hyperref}
\iftoggle{colorlinks}{%
    \hypersetup{
        citecolor = DarkGreen,
        linkcolor = DarkBlue,
        menucolor = black,
        urlcolor = DarkBlue,
    }
}{%
    \hypersetup{
        citecolor = black,
        linkcolor = black,
        menucolor = black,
        urlcolor = black,
    }
}

%%% REFERENCES
\usepackage[capitalize]{cleveref}
\crefname{defn}{definition}{definitions}

%%% TYPOGRAPHIC SETUP
\makeatletter
    % Do not hyperlink \citeauthor{}
    \pretocmd{\NAT@citexnum}{\@ifnum{\NAT@ctype>\z@}{\let\NAT@hyper@\relax}{}}{}{}
    % Make \emph{} bold in an italicized environment
    \DeclareRobustCommand\em
        {\@nomath\em \ifdim \fontdimen\@ne\font >\z@
        \bfseries \else \itshape \fi}
    \renewcommand{\@todonotes@todolistname}{List of TODOs}
\makeatother
\clubpenalty = 10000
\widowpenalty = 10000
\displaywidowpenalty = 10000

%%% THEOREM-LIKE ENVIRONMENTS
\newtheorem{theorem}{Theorem}
\newtheorem{lemma}{Lemma}
\newtheorem{defn}{Definition}
\theoremstyle{definition}
\newtheorem{example}{Example}
\newtheorem{problem}{Problem}
\newtheorem{scenario}{Scenario}

%%% MATH OPERATORS
\DeclareMathOperator*{\bigvsaunion}{\adjustbox{scale=1.6, valign=m, raise=-2pt}{$\bm{\mathsf{\cup}}$}}
\DeclareMathOperator*{\codomain}{\textbf{codom}}
\DeclareMathOperator*{\Argmin}{Arg\!\min}
\DeclareMathOperator*{\Argmax}{Arg\!\max}
\DeclareMathOperator*{\argmin}{\arg\!\min}
\DeclareMathOperator*{\argmax}{\arg\!\max}

%%%%%%%%%%%%%%%%%%%%%%%%%%%%%%%%%%%%%%%%%%%%%%%%%%%%%%%%%%%%%%%%%%
%%%                     BEGIN DOCUMENT                         %%%
%%%%%%%%%%%%%%%%%%%%%%%%%%%%%%%%%%%%%%%%%%%%%%%%%%%%%%%%%%%%%%%%%%
\begin{document}
%%% TEXTUAL VARIABLES
\newcommand{\PROSE}{\text{PROSE}\xspace}
\newcommand{\fe}{\text{FlashExtract}\xspace}
\newcommand{\ff}{\text{FlashFill}\xspace}

%%% NOTATION
\newcommand{\dsl}{\ensuremath{\mathcal{L}}}
\newcommand{\state}{\ensuremath{\sigma}}
\newcommand{\inputSymbol}[1]{\ensuremath{\mathsf{input}(#1)}}
\newcommand{\start}[1]{\ensuremath{\mathsf{output}(#1)}}
\newcommand{\rhs}[1]{\ensuremath{\mathsf{RHS}(#1)}}
\newcommand{\freeVariables}[1]{\ensuremath{\mathsf{FV}(#1)}}
\newcommand{\spec}{\ensuremath{\varphi}}
\newcommand{\constraint}{\ensuremath{\psi}}
\newcommand{\synalgorithm}{\ensuremath{\mathcal{A}}}
\newcommand{\rank}{\ensuremath{h}}
\newcommand{\join}{\ensuremath{\Join}}
\newcommand{\volume}[1]{\ensuremath{V(#1)}}
\newcommand{\width}[1]{\ensuremath{W(#1)}}
\newcommand{\vsa}{\mathcal{S}}
\newcommand{\values}[1][v]{\ensuremath{\vec{#1}}}
\newcommand{\states}[1][\state]{\ensuremath{\vec{#1}}}
\newcommand{\joinCons}[1][F]{#1_{\text{\scriptsize\join}}}
% \newcommand{\wf}[1][]{\,\mathrel{\bm{\Longrightarrow}_{\!\!#1}}\,}
% \newcommand{\wfc}[1][]{\,\mathrel{\bm{\Longleftrightarrow}_{\!\!#1}}\,}
% \newcommand{\dep}[1][]{\mathrel{\bm{\models}}}
% \newcommand{\result}{\ensuremath{\semantics{\cdot}}}
\newcommand{\atsign}{\makeatletter @\makeatother}
\newcommand{\inv}{{\scriptscriptstyle-\!1}}
\newcommand{\tospec}{\ensuremath{\rightsquigarrow}}
\newcommand{\Bool}{\ensuremath{\mathsf{Bool}}}
\newcommand{\Reals}{\ensuremath{\mathbb{R}}}
\newcommand{\true}{\ensuremath{\mathsf{True}}}
\newcommand{\false}{\ensuremath{\mathsf{False}}}
\newcommand{\ffdsl}{\ensuremath{\dsl_{\mathsf{FF}}}}
\newcommand{\fedsl}{\ensuremath{\dsl_{\mathsf{FE}}}}
\newcommandtwoopt{\disambScore}[2][q][\vsa]{\ensuremath{\mathsf{ds}(#1 , #2)}}

%%% MACROS
\newcommand{\vsajoin}[1][F]{\ensuremath{#1_{\text{\tiny\join}}}}
\newcommand{\vsaunion}{\ensuremath{\bm{\mathsf{\cup}}}}
\newcommand{\vsaunionop}{\ensuremath{\mathbin{\vsaunion}}}
\newcommand{\semantics}[1]{\ensuremath{\llbracket #1 \rrbracket}}
\newcommand{\is}{\;\coloneq\;}
\newcommand{\palt}{\;|\;}
\newcommand{\bydef}{\,\mathrel{\overset{\mathsf{\scriptscriptstyle def}}{=}}\,}
% \newcommand{\pvec}[1]{\ensuremath{\vec{#1}\mkern2mu\vphantom{#1}}}
\newcommand{\bigmid}{\mathrel{\big|}}
\newcommand{\assuming}{\;\ifnum\currentgrouptype=16 \middle\fi\vert\;}
\newcommandtwoopt{\clustering}[2][\vsa][\state]{#1 /_{#2}}
\newcommand{\todoinline}[1]{\vspace*{\topsep}\todo[inline, caption={\textbf{\textsf{TODO}}}]{#1}}

\newcommand{\stringliteral}[1]{\text{\begin{small}``\texttt{#1}''\end{small}}}
\robustify{\stringliteral}


\todototoc\listoftodos

%%% PRELIMINARY PAGES
\hypersetup{pageanchor=false}
\prelimpages
\Title{A Framework for Mass-Market Inductive Program Synthesis}
\Author{Oleksandr Polozov}
\Program{Computer Science \& Engineering}
\Year{2017}

\Chair{Sumit Gulwani}{Affiliate Professor}{Paul G. Allen School of Computer Science \& Engineering}[r]
\Chair{Zoran Popovi\'c}{Professor}{Paul G. Allen School of Computer Science \& Engineering}[.]
\Signature{Rastislav Bodik}
\Signature{Emina Torlak}

\titlepage
\newpage
\copyrightpage

\abstract{
    Programming by examples (PBE), or inductive program synthesis, is a problem of finding a program in the underlying
    domain-specific language (DSL) that is consistent with the given input-output examples or constraints.
    In the last decade, it has gained a lot of prominence thanks to the mass-market deployments of several PBE-based
    technologies for data wrangling -- the widespread problem of transforming raw datasets into a structured form, more
    amenable to analysis.
    However, deployment of a mass-market application of program synthesis is challenging.
    First, an efficient implementation requires non-trivial engineering insight, often overlooked in a research
    prototype.
    This insight takes the form of domain-specific knowledge and heuristics, which complicate the implementation,
    extensibility, and maintenance of the underlying synthesis algorithm.
    Second, application development should be fast and agile, tailoring to versatile market requirements.
    Third, the underlying synthesis algorithm should be accessible to the engineers responsible for product maintenance.
    \par
    In this work, I show how to generalize the ideas of $10$+ previous specialized inductive synthesizers into a single
    framework, which facilitates automatic generation of a domain-specific synthesizer from the mere definition of the
    corresponding DSL and its properties.
    PROSE (PROgram Synthesis using Examples) is the first program synthesis framework that explicitly separates domain-agnostic search algorithms from
    domain-specific expert insight, making the resulting synthesizer both fast and accessible.
    The underlying synthesis algorithm pioneers the use of deductive reasoning for designer-defined domain-specific
    operators and languages, which enables mean synthesis times of 1-3 sec on real-life datasets.
    \par
    A dedicated team at Microsoft has built and deployed $10$+ technologies on top of the PROSE framework.
    Using them as case studies, I examine the user interaction challenges that arise after a mass-market deployment of a PBE-powered application.
    I show how expressing program synthesis as an interactive problem facilitates user intent disambiguation,
    incremental learning from additional examples, and increases the users' confidence in the system.
}

\tableofcontents
\listoffigures
\listoftables
\hypersetup{pageanchor=true}
\acknowledgments{
    I am incredibly grateful to Sumit Gulwani, who first introduced me to the world of research, mentored and advised
    me, believed in my ideas, supported me, and filled our collaboration with indescribable energy, support, and drive.
    Sumit taught me to value simple problem solutions that leave an impact on millions of people, to persevere in the
    face of failure or rejection, and to recognize good research skills.
    Whether in work or in personal life, I knew I could always rely on Sumit's kind words and advice, and for that I
    remain eternally grateful and indebted.

    I have also been blessed to be advised by Zoran Popovi\'c.
    His vision and multi-disciplinary ambitions have always pushed me beyond my comfort zone, helped to keep a bigger
    picture in mind, and motivated to explore further.
    Working with Zoran has been a pleasure, and I will always remember his support, patience, and advice.

    I would like to thank all my co-authors, collaborators, colleagues, and friends who have helped me with these and
    other projects over the course of my graduate education:
    Erik Andersen,
    Rastislav Bodik,
    Marc Brockschmidt,
    Eric Butler,
    Sarah Chasins,
    Loris D'Antoni,
    Kevin Ellis,
    Michael Ernst,
    Elena Glassman,
    Maxim Grechkin,
    Bj\"orn Hartmann,
    Pushmeet Kohli,
    Ranvijay Kumar,
    Vu Le,
    Mark Marron,
    Mika\"el Mayer,
    Saswat Padhi,
    Daniel Perelman,
    Mark Plesko,
    Mohammad Raza,
    Eleanor O'Rourke,
    Danny Simmons,
    Rishabh Singh,
    Adam M. Smith,
    Gustavo A. Soares,
    Emina Torlak,
    Abhishek Udupa,
    Luke Zettlemoyer,
    and
    Ben Zorn.

    Last but not least, I want to thank my parents and Sasha for their unending love and support, for words of pride and
    encouragement, and for tolerating my stressful schedule and mood swings.
    Without you, none of this would ever happen.
}


%%% CONTENT
\textpages
\subimport{intro/}{main}
\chapter{Related Work}
\label{ch:related}

\subimport{background/}{main}
\subimport{vsa/}{main}
\subimport{prose/}{main}
\subimport{interaction/}{main}
\chapter{Conclusions and Future Work}
\label{ch:conclusion}


%%% BIBLIOGRAPHY
\bibliographystyle{plainnat}
\bibliography{dissertation}

%%% APPENDICES
\appendix

\end{document}

