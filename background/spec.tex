\section{Inductive Specification}
\label{sec:background:spec}

\emph{Inductive synthesis} or \emph{programming by examples} traditionally refers to the process of synthesizing a
program from a specification that consists of input-output examples.
Tools like FlashFill~\cite{flashfill}, IGOR2~\cite{igor2}, Magic Haskeller~\cite{magichaskeller}
fall in this category.
Each of them accepts a conjunction of pairs of concrete values for the input state and the corresponding output.
We generalize this formulation in two ways:
\begin{enumerate*}[label=\textbf{(\alph*)}]
    \item by extending the specification to \emph{properties} of program output as opposed to just its \emph{value}, and
        \label{item:spec:properties}
    \item by allowing arbitrary boolean connectives instead of just conjunctions. \label{item:spec:connectives}
\end{enumerate*}

Generalization \ref*{item:spec:properties} is useful when completely describing the output on a given input is too
cumbersome for a user.
For example, in FlashExtract the user provides instances of strings that should (or should not) \emph{belong} to the
output list of selections in a textual document.
Describing an entire output list would be too time-consuming and error-prone.

Generalization \ref*{item:spec:connectives} arises as part of the \emph{problem reduction} process that happens
internally in the synthesizer algorithms.
Specifications on DSL operators get refined into specifications on operator parameters, and the latter specifications
are often shaped as arbitrary boolean formulas.
For instance, in FlashFill, to synthesize a substring program that extracts a substring $s$ from a given input
string $v$, we synthesize a position extraction program that returns \emph{any} occurrence of $s$ in $v$ (which is a
\emph{disjunctive} specification).
In addition, boolean connectives may also appear in the top-level specifications provided by users.
For instance, in FlashExtract, a user may provide a \emph{negative example} (an element \emph{not} belonging to the
output list).

\begin{defn}[Inductive specification]
    Let $N\colon\tau$ be a symbol in a DSL $\dsl$.
    An \emph{inductive specification} (``spec'') $\spec$ for a program rooted at $N$ is a quantifier-free first-order
    predicate of type $\vec{\tau} \to \mathsf{Bool}$ in NNF with $n$ atoms $\langle \state_1, \constraint_1\rangle$,
    \dots, $\langle \state_{|\vec{\tau}|}, \constraint_{|\vec{\tau}|}\rangle$.
    Each atom is a pair of a concrete input state $\state_i$ over $\freeVariables{N}$ and a \emph{constraint}
    $\constraint_i\colon \tau \to \mathsf{Bool}$ constraining the output of a desired program on the input $\state_i$.
\end{defn}

\begin{defn}[Valid program]
    We say that a program $P$ \emph{satisfies} a spec $\spec$ (written $P \models \spec$) iff the formula
    $\spec\left[ \langle \state_i, \constraint_i\rangle \coloneqq \constraint_i\left(\semantics{P}{\state_i}\right)
    \right]$ holds.
    In other words, $\spec$ should hold as a boolean formula over the statements ``the output of $P$ on $\state_i$
    satisfies the constraint $\constraint_i$'' as atoms.
    % We often denote the output being constrained as $\result$, assuming implicit $P$ and $\state$.
\end{defn}

\begin{defn}[Valid program set]
    A program set $\vsa$ is \emph{valid} for spec $\spec$ (written $\vsa \models \spec$) iff all programs in $\vsa$
    satisfy $\spec$.
\end{defn}

\begin{example}
    Given a set of I/O examples $\left\{\state_i, o_i\right\}_{i=1}^m$,
    a program $P$ is valid iff $\semantics{P}{\state_i} = o_i$ for all $i$.
    In our formulation, such a spec is represented as a conjunction $\spec = \bigwedge_{i=1}^m \langle \state_i,
    \constraint_i\rangle$.
    Here each constraint $\constraint_i$ on the program output is an \emph{equality predicate}:
    $\constraint_i(v) \coloneqq [v = o_i]$.
\end{example}

\begin{example}
    In FlashExtract, a program's output is a sequence of spans $\langle l, r\rangle$ in a text document~$D$.
    A user provides a sequence of \emph{positive} examples $Y$ (a subsequence of spans from the desired
    output), and a set of \emph{negative} examples $N$ (a set of spans that the desired output should not intersect).

    The spec here is a conjunction $\spec = \langle \state, \constraint^+\rangle \wedge
    \langle\state,\constraint^-\rangle$, where positive constraint $\constraint^+$ is a \emph{subsequence predicate}:
    $\constraint^+(v) \coloneqq [v \sqsubseteq Y]$, and negative constraint $\constraint^-$ is a
    \emph{non-intersection predicate}: $\constraint^-(v) \coloneqq [v \cap N = \emptyset]$.
    The associated input states for both constraints are equal to the same $\state = \left\{doc \mapsto D\right\}$,
    where symbol $doc = \inputSymbol{\dsl}$.
    \label{ex:dsl:flashextract}
\end{example}

Often a spec $\spec$ can be decomposed into a conjunction of output properties that must be satisfied on \emph{distinct}
input states.
In this case we use a simpler notation $\spec = \left\{ \state_i \rightsquigarrow \constraint_i \right\}_{i=1}^m$.
A program~$P$ satisfies this spec if its output on each input state $\state_i$ satisfies the constraint $\constraint_i$.
For instance, I/O examples are written as $\left\{\state_i \rightsquigarrow o_i\right\}_{i=1}^m$, and
subsequence specs of FlashExtract are written as $\state \rightsquigarrow (\result \sqsubset Y) \wedge (\result
\cap N = \emptyset)$, where $\result$ denotes the desired output of $P$.

