\section{Domain-Specific Language}

\begin{figure}[p!]
    \begin{fullpage}
        \uwsinglespace
        \centering
        % \small
        \newcommand{\onep}{\textsuperscript{+}}
        \newcommand{\optional}{\textsubscript{\textsf{opt}}\ }
        \renewcommand{\synt}[1]{\textsl{#1}}
        \renewcommand{\syntleft}{\slshape}
        \renewcommand{\syntright}{}
        \begin{tcolorbox}
            \begin{grammar}
                \parskip=3pt
                <document> ::= <reference>* <using>* <languge-name> <decl>\onep

                <reference> ::= \textbf{\#reference} <dll-path> \pmb{;}

                <using> ::= \textbf{using} <namespace> \pmb{;} \\
                | \textbf{using semantics} <type> \pmb{;} \\
                | \textbf{using learners} <type> \pmb{;} \\
                | \textbf{using language} <namespace> = <member-name> \pmb{;}

                <language-name> ::= \textbf{language} <namespace> \pmb{;}

                <decl> ::= <annotation>* <type> <symbol> (\textbf{[} <params> \textbf{]})\optional (\textbf{:=} <nonterminal-body>)\optional \pmb{;}

                <annotation> ::= \textbf{@start} | \textbf{@id[}\synt{symbol-name}\textbf{]} | \textbf{@values[}\synt{member-name}\textbf{]} | \dots

                <params> ::= <param> (\pmb{,} <param>)*

                <param> ::= <type> <symbol>

                <nonterminal-body> ::= <rule> (\pmb{|} <rule>)*

                <rule> ::= (<namespace> \textbf{.})\optional <symbol> \\
                | (<namespace> \textbf{.})\optional <operator-name>$\pmb{\bigl(}$(<arg> (\pmb{,} <arg>)* )\optional$\pmb{\bigr)}$ \\
                | \textbf{let} <type> <symbol> \textbf{=} <rule> \textbf{in} <rule>

                <arg> ::= <symbol> | $\bm{\lambda}$ <symbol> $\bm{\Rightarrow}$ <rule>

                <symbol,~operator-name,~namespace> ::= $\langle$\textup{id}$\rangle$

                <dll-path,~symbol-name> ::= \pmb{\textquotesingle}$\langle$\textup{string}$\rangle$\pmb{\textquotesingle}

                <member-name> ::= $\langle$\textup{member in a target language}$\rangle$

                <type> ::=  $\langle$\textup{type in a target language}$\rangle$
            \end{grammar}
        \end{tcolorbox}
        \caption{PROSE DSL definition language.
            A DSL is a set of (typed and annotated) symbol definitions, where each symbol is either a terminal, or a
            nonterminal (possibly with parameters) defined through a set of rules.
            Each rule is a conversion of nonterminals, an operator with some arguments (symbols or $\lambda$-functions), or
            a \texttt{let} definition.
            A DSL may reference types and members from external libraries, as well as other DSLs (treated as namespace
            imports).
        }
        \label{fig:dsldefinition}
    \end{fullpage}
\end{figure}


A synthesis problem is defined for a given \emph{domain-specific language} (DSL) $\dsl$.
The language of DSL definitions is given in \Cref{fig:dsldefinition}.
A DSL is specified as a context-free grammar (CFG), with each nonterminal symbol $N$ defined through a set of
\emph{rules}.
Each rule has on its right-hand side an application of an \emph{operator} to some symbols of $\dsl$, and we denote the
set of all possible operators on the right-hand sides of the rules for $N$ as $\rhs{N}$.
All symbols and operators are \emph{typed}.
Every symbol $N$ in the CFG is annotated with a corresponding output type $\tau$, denoted $N\colon \tau$.
If $N := F(N_1, \dots, N_k)$ is a grammar rule and $N\colon \tau$, then the output type of $F$ must be $\tau$.
A DSL has a designated \emph{start symbol} $\start{\dsl}$, which is a root nonterminal in the CFG of $\dsl$.

Every (sub-)program $P$ rooted at a symbol $N\colon \tau$ in $\dsl$ maps an \emph{input state}\footnote{DSLs in PBE
    typically do not involve mutation, so an input $\state$ is technically an \emph{environment}, not a \emph{state}.
    We keep the term ``state'' for historical reasons.} $\state$ to a value of type $\tau$.
The execution result of a program $P$ on a state $\state$ is written as $\semantics{P}{\state}$.
A state is a mapping of free variables $\freeVariables{P}$ to their bound values.
Variables in a DSL are either explicitly declared as \emph{nonterminal parameters}, or introduced by \texttt{let}
definitions and $\lambda$-functions.
The start symbol has a single parameter variable --- an \emph{input symbol} $\inputSymbol{\dsl}$ of the DSL.

Every operator $F$ in a DSL $\dsl$ has some executable semantics.
Many operators are generic, included in the standard library (\texttt{std}), and typically reused across different DSLs
(e.g. $\mathsf{Filter}$ and $\mathsf{Map}$ list combinators).
Others are domain-specific, and defined only for a given DSL.
Operators are assumed to be deterministic and pure, modulo unobservable side effects.

Symbols and rules may be augmented with multiple custom annotations, such as:
\begin{itemize}[nosep]
    \item \dslinline|@start| -- marks the start symbol of a DSL;
    \item \dslinline|@id['$\text{\rmfamily\slshape\color{Brown} symbol-name}$']| -- gives a designated name to the
        following rule, which may be used to reference it from the accompanying source code;
    \item \dslinline|@values[$\text{\rmfamily\slshape member-name}$]| -- specifies a member (e.g. a field, a static
        variable, a property) in a target language that stores a set of possible values for the given terminal.
\end{itemize}

