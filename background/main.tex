\chapter{Background}
\label{ch:background}
This section presents the preliminaries that are used throughout the thesis.
First, I introduce FlashFill~\cite{flashfill} and FlashExtract \cite{flashextract}, two instances of prior
synthesis work that are generalized by the PROSE framework.
Then, I use them to describe the notation for PROSE DSLs and illustrate it on real-life motivating examples.

\section{Programming by Examples}
\label{sec:background:pbe}

\Cref{fig:dsl:flashfill} shows definitions of FlashFill and FlashExtract in the input syntax of PROSE.

\paragraph{FlashFill}
FlashFill is a system for synthesis of string transformations in Microsoft Excel spreadsheets from input-output examples.
Each program $P$ in the FlashFill DSL $\ffdsl$ takes as input a tuple of \emph{user input strings} $v_1, \dots, v_k$,
and returns an output string.

The FlashFill language $\ffdsl$ is structured in three layers.
On the topmost layer (omitted in \Cref{fig:dsl:flashfill}), an $n$-ary
\textsf{Switch} operator evaluates $n$ predicates over the input string tuple, and chooses the branch that then produces
the output string.
\todoinline{Put \textsf{Switch} in the figure}
Each \textsf{Switch} branch is a \emph{concatenation} of some number of \emph{atomic string expressions}, which
produce pieces of the output string.
Each such atomic expression can either be a \emph{constant}, or a \emph{substring} of one of the input strings $v_1,
\dots, v_k$.
The \emph{position expression} logic for choosing starting and ending positions for the substring operator can either be
\emph{absolute} (e.g.  ``$5^{\text{th}}$ character from the right''), or based on \emph{regular expression matching}
(e.g. ``the last occurrence of a number'').

\begin{figure*}
    \centering
    \lstset{basicstyle = \footnotesize\ttfamily}
    \def\baselinestretch{1.05}
    \begin{subfigure}{1.01\textwidth}
    \hfill\textbf{(a)}\vspace{-0.8\baselineskip}
    \begin{lstlisting}[language=dsl,gobble=8,morekeywords={Regex}]
        language FlashFill;
        // Nonterminals
        @start string $transform$[string[] $inputs$] $\coloneq$ $atom$ | Concat($atom$, $transform$[$inputs$]);
        string $atom$ $\coloneq$ ConstStr($s$) | let string $x$ = std.Kth($inputs$, $k$) in Substring($x$, $pp[x]$);
        Tuple<int, int> $pp$[string $x$] $\coloneq$ std.Pair($pos[x]$, $pos[x]$);
        int $pos$[string $x$] $\coloneq$ AbsPos($x$, $k$) | RegexPos($x$, $rr$, $k$);
        Tuple<Regex, Regex> $rr$ $\coloneq$ std.Pair($r$, $r$);
        // Terminals
        string $s$;  $\qquad$   int $k$;  $\qquad$  Regex r;
    \end{lstlisting}
    \end{subfigure}
    \begin{subfigure}{\textwidth}
    \hrule
    \vspace{5pt}
    \hfill\textbf{(b)}
    \vspace{-0.9\baselineskip}
    \begin{lstlisting}[language=csharp,gobble=8,breaklines=false,morekeywords={Tuple}]
        string Concat(string atom, string transform) => atom + transform;
        string ConstStr(string s) => s;
        string Substring(string x, Tuple<int, int> pp) {
            int l = pp.Item1, r = pp.Item2;
            return (l < 0 || r > x.Length) ? null : x.Substring(l, r - l);
        }
        int AbsPos(string x, int k) => k < 0 ? x.Length + k + 1 : k;
        int? RegexPos(string x, Tuple<Regex, Regex> rr, int k) {
            Regex r = new Regex("(?<=" + rr.Item1 + ")" + rr.Item2);
            MatchCollection ms = r.Matches(x);
            int i = k > 0 ? (k - 1) : (k + ms.Count);
            return (i < 0 || i >= ms.Count) ? null : ms[i].Index;
        }
        [Values("r")] static readonly Regex[] StaticTokens = {
            new Regex("\\d+"), new Regex("[a-z]+"),  /* more tokens... */
        }
    \end{lstlisting}
    \end{subfigure}
    \begin{subfigure}{\textwidth}
    \hrule
    \vspace{5pt}
    \hfill\textbf{(c)}
    \vspace{-0.9\baselineskip}
    % \begin{lstlisting}[language=csharp,gobble=8]
    %     Grammar ff = new Grammar("FlashFill.grammar");
    %     ProgramNode p = ff.ParseAST("let x = std.Kth(v, 0) in
    %         Substring(x, RegexPos(x, std.Pair(new Regex(\"\"), new Regex(@\"[A-Z]+\")), -1),
    %                    AbsPos(-1))");
    %     State input = new State(ff.Symbol("vs"), new string[] {"Leslie Lamport"});
    %     Assert.Equals(p.Invoke(input), "Lamport");
    % \end{lstlisting}
    % \end{subfigure}
    % \begin{subfigure}{\textwidth}
    % \hrule
    % \vspace{5pt}
    % \hfill\textbf{(d)}
    % \vspace{-1.3\baselineskip}
    \begin{lstlisting}[language=dsl,gobble=8,morekeywords={StringRegion,Regex}]
        language FlashExtract.Text;
        using grammar FF = FlashFill.Language;
        // Nonterminals
        @start StringRegion[] $seq$[StringRegion $doc$] $\coloneq$
              @id['LinesMap'] std.Map($\kwlambda\, x \kwfun$ std.Pair($pos$[$x$], $pos$[$x$]), $lines$)
            | @id['StartSeqMap'] std.Map($\kwlambda\, p \kwfun$ std.Pair($p$, $pos$[GetSuffix($doc$, $p$)]), $positions$)
            | @id['EndSeqMap'] std.Map($\kwlambda\, p \kwfun$ std.Pair($pos$[GetPrefix($doc$, $p$)], $p$), $positions$);
        int[] $positions$ $\coloneq$ std.FilterInt($init$, $step$, $regexPositions$);
        int[] $regexPositions$ $\coloneq$ RegexMatches($doc$, std.Pair($r$, $r$));
        int $pos$[string $x$] $\coloneq$ FF.$pos$[$x$];  // External nonterminal
        StringRegion[] $lines$ $\coloneq$ std.FilterInt($init$, $step$, $filterLines$);
        StringRegion[] $filterLines$ $\coloneq$ std.Filter($\kwlambda\, \ell \!\kwfun\, match$, $docLines$);
        StringRegion[] $docLines$ $\coloneq$ SplitLines($doc$);
        bool $match$ $\coloneq$ Contains($\ell$, $r$) | StartsWith($\ell$, $r$) | EndsWith($\ell$, $r$);
        // Terminals
        int $init$;   $\qquad$  int $step$;  $\qquad$  Regex $r$;
    \end{lstlisting}
    \end{subfigure}
    \caption{\textbf{(a)} A DSL of FlashFill substring extraction $\ffdsl$.
    \textbf{(b)} Executable semantics of FlashFill operators, defined by the DSL designer in C\#, and
    a set of possible values for the terminal~$r$.
    \textbf{(c)}
    % Example program usage in $\ffdsl$. The program selects the last capitalized word (a substring starting from the last
    % capital letter) from the first input string $v_1$.
    % \textbf{(d)}
    FlashExtract DSL $\fedsl$ for selection of spans in a textual document $doc$.
    It references position extraction logic $pos$ from $\ffdsl$. }
    \label{fig:dsl:flashfill}
\end{figure*}

\begin{example}[{Adapted from \cite[Example 10]{flashfill}}]
    \label{ex:background:ff}
    Consider the problem of formatting phone numbers in a list of attendees:
    \begin{center}
        \small
        \begin{tabular}{lll}
            \toprule
            % \textbf{Input $\bm{v_1}$} & \textbf{Input $\bm{v_2}$} & \textbf{Output} \\
            \midrule
            323-708-7700 & Dr. Leslie B. Lamport & (323) 708-7700 \\
            (425)-706-7709 & Bill Gates, Sr. & (425) 706-7709 \\
            510.220.5586 & George Ciprian Necula & (510) 220-5586 \\
            \bottomrule
        \end{tabular}
    \end{center}
    One possible FlashFill program that performs this task is shown below ($\circ$ denotes concatenation):

    \begin{multline*}
        \mathsf{ConstStr}(\stringliteral{(})
        \, \circ\, \mathsf{Match}(v_1,\! \stringliteral{\textbackslash d+}, 1) \circ\,
        \mathsf{ConstStr}(\stringliteral{)\textvisiblespace})
        \,\circ\, \\ \mathsf{Match}(v_1,\! \stringliteral{\textbackslash d+}, 2) \,\circ\,
        \mathsf{ConstStr}(\stringliteral{-}) \,\circ\, \mathsf{Match}(v_1,\! \stringliteral{\textbackslash d+}, 3)
    \end{multline*}
    where $\mathsf{Match}(x, r, k)$ is a $k^{\text{th}}$ match of a regex $r$ in $x$.
    In $\ffdsl$, $\mathsf{Match}(x, r, k)$ is an abbreviation for
    ``$\text{\ttfamily let } x = v_1$ \texttt{in } $\mathsf{SubStr}(x,\langle \mathsf{RegPos}(x, \langle \varepsilon, r\rangle, k),
    \mathsf{RegPos}(x, \langle r, \varepsilon\rangle, k)\rangle)$''. \linebreak
    We also use the notation $f_1 \circ \dots \circ f_n$ for $n$-ary string concatenation:
    ``$\mathsf{Concat}(f_1,\, \mathsf{Concat}(\dots, \mathsf{Concat}(f_{n-1},\, f_n)))$''.
\end{example}
