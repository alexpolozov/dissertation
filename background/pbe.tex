\section{Programming by Examples}
\label{sec:background:pbe}

As discussed in \Cref{ch:related}, the field of domain-specific programming by examples (PBE) started with
FlashFill~\cite{flashfill} and its successors: FlashExtract~\cite{flashextract}, FlashRelate~\cite{flashrelate}, and
others.
In this work, I use FlashFill and FlashExtract as the running examples of PBE DSLs generalized by the PROSE framework.
For future reference, \Cref{fig:dsl:flashfill} shows definitions of FlashFill and FlashExtract in the input syntax of
PROSE.

\paragraph{FlashFill}
FlashFill is a system for synthesis of string transformations in Microsoft Excel spreadsheets from input-output
examples.
Each program $P$ in the FlashFill DSL $\ffdsl$ takes as input a tuple of \emph{user input strings} $v_1, \dots, v_k$,
and returns an output string.

\todoinline{Mark all \textbf{(a)}, \textbf{(b)}, \dots\ as subfigures with hyperlinks.}
The FlashFill language $\ffdsl$ is structured in three layers.
On the topmost layer, an $n$-ary \textsf{ITE} (``if-then-else'') operator evaluates $n$ predicates over the input string
tuple, and chooses the branch that then produces the output string.
Each \textsf{ITE} branch is a \emph{concatenation} of some number of \emph{atomic string expressions}, which
produce pieces of the output string.
Each such atomic expression can either be a \emph{constant}, or a \emph{substring} of one of the input strings $v_1,
\dots, v_k$.
The \emph{position expression} logic for choosing starting and ending positions for the substring operator can either be
\emph{absolute} (e.g.  ``$5^{\text{th}}$ character from the right''), or based on \emph{regular expression matching}
(e.g. ``the last occurrence of a number'').

\begin{figure}[p!]
    \begin{fullpage}
        \centering
        \lstset{basicstyle = \footnotesize\ttfamily}
        \def\baselinestretch{1.05}
        \hfill\textbf{(a)}
        \vspace{-0.8\baselineskip}
        \begin{lstlisting}[language=dsl,gobble=12,morekeywords={Regex}]
            language FlashFill;
            // Nonterminals
            @start bool $expr$[string[] $inputs$] $\coloneq$
                let string $\ell$ = std.Kth($inputs$, $k$) in std.ITE($match$[$\ell$], $transform$, $expr$[$inputs$]);
            string $transform$ $\coloneq$ $atom$ | Concat($atom$, $transform$);
            string $atom$ $\coloneq$ ConstStr($s$) | let string $x$ = std.Kth($inputs$, $k$) in Substring($x$, $pp[x]$);
            Tuple<int, int> $pp$[string $x$] $\coloneq$ std.Pair($pos[x]$, $pos[x]$);
            int $pos$[string $x$] $\coloneq$ AbsPos($x$, $k$) | RegexPos($x$, std.Pair($r$, $r$), $k$);
            bool $match$[string $\ell$] $\coloneq$ Contains($\ell$, $r$) | StartsWith($\ell$, $r$) | EndsWith($\ell$, $r$);
            // Terminals
            string $s$;  $\qquad$   int $k$;  $\qquad$  Regex r;
        \end{lstlisting}
        \hrule
        \vspace{5pt}
        \hfill\textbf{(b)}
        \vspace{-0.9\baselineskip}
        \begin{lstlisting}[language=csharp,gobble=12,breaklines=false,morekeywords={Tuple}]
            string Concat(string atom, string transform) => atom + transform;
            string ConstStr(string s) => s;
            string Substring(string x, Tuple<int, int> pp) {
                int l = pp.Item1, r = pp.Item2;
                return (l < 0 || r > x.Length) ? null : x.Substring(l, r - l);
            }
            int AbsPos(string x, int k) => k < 0 ? x.Length + k + 1 : k;
            int? RegexPos(string x, Tuple<Regex, Regex> rr, int k) {
                Regex r = new Regex("(?<=" + rr.Item1 + ")" + rr.Item2);
                MatchCollection ms = r.Matches(x);
                int i = k > 0 ? (k - 1) : (k + ms.Count);
                return (i < 0 || i >= ms.Count) ? null : ms[i].Index;
            }
            [Values("r")] static readonly Regex[] StaticTokens =
                { new Regex("\\d+"), new Regex("[a-z]+"),  /* more tokens... */ }
        \end{lstlisting}
        \hrule
        \vspace{5pt}
        \hfill\textbf{(c)}
        \vspace{-0.9\baselineskip}
        % \begin{lstlisting}[language=csharp,gobble=12]
        %     Grammar ff = new Grammar("FlashFill.grammar");
        %     ProgramNode p = ff.ParseAST("let x = std.Kth(v, 0) in
        %         Substring(x, RegexPos(x, std.Pair(new Regex(\"\"), new Regex(@\"[A-Z]+\")), -1),
        %                    AbsPos(-1))");
        %     State input = new State(ff.Symbol("vs"), new string[] {"Leslie Lamport"});
        %     Assert.Equals(p.Invoke(input), "Lamport");
        % \end{lstlisting}
        % \end{subfigure}
        % \begin{subfigure}{\textwidth}
        % \hrule
        % \vspace{5pt}
        % \hfill\textbf{(d)}
        % \vspace{-1.3\baselineskip}
        \begin{lstlisting}[language=dsl,gobble=12,morekeywords={StringRegion,Regex}]
            language FlashExtract.Text;
            using grammar FF = FlashFill.Language;
            // Nonterminals
            @start StringRegion[] $seq$[StringRegion $doc$] $\coloneq$
                  @id['LinesMap'] std.Map($\kwlambda\, x \kwfun$ std.Pair($pos$[$x$], $pos$[$x$]), $lines$)
                | @id['StartSeqMap'] std.Map($\kwlambda\, p \kwfun$ std.Pair($p$, $pos$[GetSuffix($doc$, $p$)]), $positions$)
                | @id['EndSeqMap'] std.Map($\kwlambda\, p \kwfun$ std.Pair($pos$[GetPrefix($doc$, $p$)], $p$), $positions$);
            int[] $positions$ $\coloneq$ std.FilterInt($init$, $step$, $regexPositions$);
            int[] $regexPositions$ $\coloneq$ RegexMatches($doc$, std.Pair($r$, $r$));
            int $pos$[string $x$] $\coloneq$ FF.$pos$[$x$];  // External nonterminal
            StringRegion[] $lines$ $\coloneq$ std.FilterInt($init$, $step$, $filterLines$);
            StringRegion[] $filterLines$ $\coloneq$ std.Filter($\kwlambda\, \ell \!\kwfun\ $FF.$match$[$\ell$], $docLines$); // External nonterminal
            StringRegion[] $docLines$ $\coloneq$ SplitLines($doc$);
            // Terminals
            int $init$;   $\qquad$  int $step$;  $\qquad$  Regex $r$;
        \end{lstlisting}
        \caption{\textbf{(a)} A DSL of FlashFill substring extraction $\ffdsl$.
            \textbf{(b)} Executable semantics of FlashFill operators, defined by the DSL designer in C\#, and
            a set of possible values for the terminal~$r$.
            \textbf{(c)}
            FlashExtract DSL $\fedsl$ for selection of spans in a textual document $doc$.
        It references position extraction logic $pos$ and Boolean predicates $match$ from $\ffdsl$. }
        \label{fig:dsl:flashfill}
    \end{fullpage}
\end{figure}

\def\wspace{\text{\color{Brown}\textvisiblespace}}
\begin{example}[{Adapted from \cite[Example 10]{flashfill}}]
    \label{ex:background:ff}
    Consider the problem of formatting phone numbers in a list of conference attendees:
    \begin{center}
        \small
        \begin{tabular}{lll}
            \toprule
            \textbf{Input} \bmsub v1 & \textbf{Input} \bmsub v2 & \textbf{Output} \\
            \midrule
            323-708-7700 & Dr. Leslie B. Lamport & (323) 708-7700 \\
            (425)-706-7709 & Bill Gates, Sr. & (425) 706-7709 \\
            510.220.5586 & George Ciprian Necula & (510) 220-5586 \\
            \bottomrule
        \end{tabular}
    \end{center}
    One possible FlashFill program that performs this task is shown below:
    \begin{lstlisting}[language=dsl]
        ConstStr('(') $\concat$ Match($v_1$, '\d+', 1) $\concat$ ConstStr(')$\wspace$') $\concat$
            Match($v_1$, '\d+', 2) $\concat$ ConstStr('-') $\concat$ Match($v_1$, '\d+', 3)
    \end{lstlisting}
    where \dslinline|Match($w$, $r$, $k$)| in $\ffdsl$ is a $k^{\text{th}}$ match of a regex $r$ in $w$ ---
    an abbreviation for
    ``\dslinline|let $x = w$ in Substring($x$, $\langle$RegexPos($x$, $\langle\varepsilon,\, r\rangle$, $k$), RegexPos($x$, $\langle r,\, \varepsilon\rangle$, $k$)$\rangle$)|''.
    Here we also use the notation $\langle x, y \rangle$ for ``\dslinline|std.Pair($x$, $y$)|'' and $f_1 \concat \dots \concat
    f_n$ for $n$-ary string concatenation:
    ``\dslinline|Concat($f_1$, Concat($\dots$, Concat($f_{n-1}$, $f_n$)))|''.
\end{example}

\paragraph{FlashExtract}
FlashExtract is a system for synthesis of scripts for extracting data from semi-structured documents.
It has been integrated in Microsoft PowerShell 3.0 for release with Windows 10 and in the Azure Operational Management
Suite for extracting custom fields from log files.
In the original publication, \citeauthor*{flashextract} present three instantiations of FlashExtract; in this work, we
focus on the \emph{text} instantiation as a running example, although \emph{Web} and \emph{table} instantiations were
also reimplemented on top of the PROSE framework (see \Cref{sec:prose:evaluation:casestudies}).

Each program in the FlashExtract DSL $\fedsl$ takes as input a \emph{textual document} $doc$, and returns a sequence
of spans in that document.
The sequence is selected based on combinations of \texttt{Map} and \texttt{Filter} operators, applied to sequences of
matches of various regexes in $doc$.

\begin{sidewaysfigure}[p!]
    \centering
    \uwsinglespace
    \begin{tcbraster}[beamer, raster columns=1, size=minimal]
        \tcbincludegraphics{figures/Playground}
    \end{tcbraster}
    \caption{An illustrative scenario for data extraction from semi-structured text using FlashExtract.}
    \label{fig:example:flashextract}
\end{sidewaysfigure}

\begin{example}
    \label{ex:background:fe}
    Consider the textual file shown in \Cref{fig:example:flashextract}.
    One possible $\fedsl$ program that extracts a sequence of yellow regions (the scores) from it is
    \begin{lstlisting}[language=dsl, gobble=8]
        LinesMap($\fun{x}$ $\langle$RegexPos($x$, $\langle$'\|', $\varepsilon\rangle$, -1), RegexPos($x$, $\langle$'\d+', $\varepsilon\rangle$, -1)$\rangle$,
            FilterInt(1, 3, Filter($\fun{\ell}$
                StartsWith($\ell$, '"|$\wspace$style="text-align:$\wspace$center;"|$\wspace${$\wspace${$\wspace$Sort|"' $\concat$ Alphanumeric),
                SplitLines($doc$))))
    \end{lstlisting}
    It splits $doc$ into a sequence of lines, selecting only the lines that start with a certain constant string
    followed by an \texttt{Alphanumeric} token.
    Then, from every third such line starting from the second one, it extracts the string between the last occurrence of
    \stringliteral{|} and the last occurrence of a number.
\end{example}

