\begin{figure}[t]
    \centering
    % \small
    % \setlength{\grammarindent}{0.5cm}
    \renewcommand{\synt}[1]{\textsl{#1}}
    \renewcommand{\syntleft}{\slshape}
    \renewcommand{\syntright}{}
    \begin{mdframed}
    \begin{grammar}
        % \parskip=1pt
        <decl> ::= <annotation>* <type> <symbol> (\textbf{:=} <nonterminal-body>)? \pmb{;}

        <annotation> ::= \textbf{@ input} | \textbf{@ output} | \textbf{@ extern[}\synt{namespace}\textbf{]} \\
        | \textbf{@ values[}\synt{member-name}\textbf{]} | \dots

        <nonterminal-body> ::= <rule> (\pmb{|} <rule>)*

        <rule> ::= <symbol> \\
        | (<namespace> \textbf{.})? <operator-name>$\pmb{\bigl(}$(<arg> (\textbf{,} <arg>)* )?$\pmb{\bigr)}$ \\
        | \textbf{let} <type> <symbol> \textbf{=} <rule> \textbf{in} <rule>

        <arg> ::= <symbol> | $\bm{\lambda}$<symbol>\textbf{:} <type> $\bm{\Rightarrow}$ <rule>

        <symbol,~operator-name,~namespace> ::= $\langle$\textup{id}$\rangle$

        <member-name> ::= $\langle$\textup{member in a target language}$\rangle$

        <type> ::=  $\langle$\textup{type in a target language}$\rangle$
    \end{grammar}
    \end{mdframed}
    \caption{PROSE DSL definition language. A DSL is a set of (typed and annotated) symbol definitions, where each
        symbol is either a terminal, or a nonterminal defined through a set of rules. Each rule is a conversion of
        nonterminals, an operator with some arguments (symbols or $\lambda$-functions), or a \texttt{let} definition.
        Some auxiliary instructions are omitted for brevity, such as namespace imports or library references.}
    \label{fig:dsldefinition}
\end{figure}
