\begin{figure}[p!]
    \begin{fullpage}
        \uwsinglespace
        \centering
        % \small
        \newcommand{\onep}{\textsuperscript{+}}
        \newcommand{\optional}{\textsubscript{\textsf{opt}}\ }
        \renewcommand{\synt}[1]{\textsl{#1}}
        \renewcommand{\syntleft}{\slshape}
        \renewcommand{\syntright}{}
        \begin{tcolorbox}
            \begin{grammar}
                \parskip=3pt
                <document> ::= <reference>* <using>* <languge-name> <decl>\onep

                <reference> ::= \textbf{\#reference} <dll-path> \pmb{;}

                <using> ::= \textbf{using} <namespace> \pmb{;} \\
                | \textbf{using semantics} <type> \pmb{;} \\
                | \textbf{using learners} <type> \pmb{;} \\
                | \textbf{using language} <namespace> = <member-name> \pmb{;}

                <language-name> ::= \textbf{language} <namespace> \pmb{;}

                <decl> ::= <annotation>* <type> <symbol> (\textbf{[} <params> \textbf{]})\optional (\textbf{:=} <nonterminal-body>)\optional \pmb{;}

                <annotation> ::= \textbf{@start} | \textbf{@id[}\synt{symbol-name}\textbf{]} | \textbf{@values[}\synt{member-name}\textbf{]} | \dots

                <params> ::= <param> (\pmb{,} <param>)*

                <param> ::= <type> <symbol>

                <nonterminal-body> ::= <rule> (\pmb{|} <rule>)*

                <rule> ::= (<namespace> \textbf{.})\optional <symbol> \\
                | (<namespace> \textbf{.})\optional <operator-name>$\pmb{\bigl(}$(<arg> (\pmb{,} <arg>)* )\optional$\pmb{\bigr)}$ \\
                | \textbf{let} <type> <symbol> \textbf{=} <rule> \textbf{in} <rule>

                <arg> ::= <symbol> | $\bm{\lambda}$ <symbol> $\bm{\Rightarrow}$ <rule>

                <symbol,~operator-name,~namespace> ::= $\langle$\textup{id}$\rangle$

                <dll-path,~symbol-name> ::= \pmb{\textquotesingle}$\langle$\textup{string}$\rangle$\pmb{\textquotesingle}

                <member-name> ::= $\langle$\textup{member in a target language}$\rangle$

                <type> ::=  $\langle$\textup{type in a target language}$\rangle$
            \end{grammar}
        \end{tcolorbox}
        \caption{PROSE DSL definition language.
            A DSL is a set of (typed and annotated) symbol definitions, where each symbol is either a terminal, or a
            nonterminal (possibly with parameters) defined through a set of rules.
            Each rule is a conversion of nonterminals, an operator with some arguments (symbols or $\lambda$-functions), or
            a \texttt{let} definition.
            A DSL may reference types and members from external libraries, as well as other DSLs (treated as namespace
            imports).
        }
        \label{fig:dsldefinition}
    \end{fullpage}
\end{figure}
