\abstract{
    Programming by examples (PBE), or inductive program synthesis, is a problem of finding a program in the underlying
    domain-specific language (DSL) that is consistent with the given input-output examples or constraints.
    In the last decade, it has gained a lot of prominence thanks to the mass-market deployments of several PBE-based
    technologies for data wrangling -- the widespread problem of transforming raw datasets into a structured form, more
    amenable to analysis.
    However, deployment of a mass-market application of program synthesis is challenging.
    First, an efficient implementation requires non-trivial engineering insight, often overlooked in a research
    prototype.
    This insight takes the form of domain-specific knowledge and heuristics, which complicate the implementation,
    extensibility, and maintenance of the underlying synthesis algorithm.
    Second, application development should be fast and agile, tailoring to versatile market requirements.
    Third, the underlying synthesis algorithm should be accessible to the engineers responsible for product maintenance.
    \par
    In this work, I show how to generalize the ideas of 10+ previous specialized inductive synthesizers into a single
    framework, which facilitates automatic generation of a domain-specific synthesizer from the mere definition of the
    corresponding DSL and its properties.
    It is the first program synthesis framework that explicitly separates domain-agnostic search algorithms from
    domain-specific expert insight, making the resulting synthesizer both fast and accessible.
    The underlying synthesis algorithm pioneers the use of deductive reasoning for designer-defined domain-specific
    operators and languages, which features mean synthesis times of 1-3 sec on real-life datasets.
    \par
    In addition, I study the user interaction challenges that arise after a mass-market deployment of a PBE-powered
    application, using 10+ technologies built and deployed by Microsoft on top of the developed synthesis framework as
    case studies.
    I show how expressing program synthesis as an interactive problem facilitates user intent disambiguation,
    incremental learning from additional examples, and increases the users' confidence in the system.
}
